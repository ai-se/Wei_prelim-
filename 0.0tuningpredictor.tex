% \documentclass[final,twocolumn,5p]{elsarticle}
% \documentclass{sig-alternative}
\documentclass[conference]{IEEEtran}
% \documentclass[smallextended]{svjour3}
% \documentclass[preprint,12pt,3p,number]{elsarticle}
\usepackage{cite}
\usepackage{multirow}
\usepackage{color}
\usepackage{graphics} 
\usepackage{rotating}
\usepackage{eqparbox}
\usepackage{graphics}
\usepackage{colortbl} 
 \usepackage{mathptmx} \usepackage[scaled=.90]{helvet} \usepackage{courier}
\usepackage{balance}
\usepackage{picture}
\usepackage{algorithm}
\usepackage{algorithmicx}
\usepackage{algpseudocode}
\usepackage[export]{adjustbox}
\renewcommand{\footnotesize}{\scriptsize}
\definecolor{lightgray}{gray}{0.8}
\definecolor{darkgray}{gray}{0.6}
\renewcommand{\algorithmicrequire}{\textbf{Input:}}
\renewcommand{\algorithmicensure}{\textbf{Output:}}
\newcommand{\crule}[3][darkgray]{\textcolor{#1}{\rule{#2}{#3}}}

\newcommand{\quart}[3]{\begin{picture}(100,6)%1
{\color{black}\put(#3,3){\circle*{4}}\put(#1,3){\line(1,0){#2}}}\end{picture}}
\definecolor{Gray}{gray}{0.95}
\definecolor{LightGray}{gray}{0.975}
\newcommand{\wei}[1]{\textcolor{red}{Wei: #1}} 
\newcommand{\Menzies}[1]{\textcolor{red}{Dr.Menzies: #1}} 

%% timm tricks
\newcommand{\bi}{\begin{itemize}[leftmargin=0.4cm]}
\newcommand{\ei}{\end{itemize}}
\newcommand{\be}{\begin{enumerate}}
\newcommand{\ee}{\end{enumerate}}
\newcommand{\tion}[1]{\S\ref{sect:#1}}
\newcommand{\fig}[1]{Figure~\ref{fig:#1}}
\newcommand{\tab}[1]{Table~\ref{tab:#1}}
\newcommand{\eq}[1]{Equation~\ref{eq:#1}}

%% space saving measures

\usepackage[shortlabels]{enumitem}  
\usepackage{url}
\begin{document}
% \begin{frontmatter}
\title{ Tuning for Software Analytics: is it Really Necessary?}
\author{Wei Fu }
\maketitle
\thispagestyle{plain}
\pagestyle{plain}

\begin{abstract}
Data mining algorithms have been widely used in software engineering to perform software 
analytics, like defect
prediction, effort estimation and text mining. But one of the “black arts” of data mining is setting
the tunings that control the miner. Nevertheless, we rarely tuned those parameters since we
reasoned that a data miner’s default tunings have been well-explored by the developers of
those algorithms.

In this work, taking defect prediction as an example, we investigated how parameter tuning can
affect the results of software analytics. For each experiment with different data sets (from open
source JAVA systems), we ran differential evolution as an optimizer to explore the tuning space
(as a first step) then tested the tunings.

Contrary to our prior expectations, we found these tunings were remarkably simple: it only
required tens, not thousands, of attempts to obtain very good results. For example, when
learning software defect predictors, this method can quickly find tunings that alter detection
precision from 0\% to 60\%.

Since the (1) the improvements are so large, and (2)the tuning is so simple, we need to change
standard methods in software analytics. At least for defect prediction, it is no longer enough to
just run a data miner and present the result.

\end{abstract}

\vspace{1mm}
\noindent
{\bf Keywords:} defect prediction, CART, random forest,
differential evolution,
search-based software engineering.
%  \maketitle 
\pagenumbering{arabic} %XXX delete before submission

\section{Introduction}

\subsection{Background}
Recent years, Parameter tuning has been attracting more attentions in fields like evolutionary
computation, data mining and  machine learning \cite{lobo2007parameter,Bergstra2012}, where the behavior of the algorithms will be controlled by the built-in parameters.  Practitioners and researchers always encounter questions like can we find an optimal set of parameters once and for all problems. Unfortunately, No Free Lunch theorem \cite{wolpert1997no} will say “NO” to this question because all algorithms perform equally well on all possible problems on average. Different sets of parameters will change the results of the algorithms in the same setting.

For evolutionary algorithms(EA) with parameters like the population size and the probability of mutation, it was realized that in order to achieve optimal convergence, these parameters should be adjusted over time by taking into account information about the progress achieved. Generally, there are two major forms of setting parameter values in EA: {\it parameter tuning} and {\it parameter control}\cite{brest2006self}. The former means the commonly practiced approach that tries to find good values for the parameters before running the algorithm and then run the algorithm using these values, which remain fixed during the run. The latter means that tuning values for the parameters happen during the run of algorithms. Due the nature of EA, where a run of an EA is an intrinsically dynamic, the use of rigid parameters that do not change their values is thus in contrast to this spirit\cite{eiben1999parameter}. Therefore,  the latter form are favored a lot. There are three categories of parameter control\cite{eiben1999parameter}:
\bi
\item {\it Deterministic parameter control} is changing the value of parameters based on predefined rules\cite{Fogarty1989}.
\item {\it Adaptive parameter control} is using the feedback from the search to determine the direction of the change\cite{schraudolph1992dynamic,shaefer1987argot}.
\item {\it Self-adaptive parameter control} is to incorporate parameters into the ``chromosomes''(a terminology in EA), thereby making them subject to evolution\cite{brest2006self,qin2005self,omran2005self,yang2008self}.
\ei.
Evolutionary algorithms have been widely applied in software engineering, eventually forming a research field called search-based software engineering(SBSE)\cite{harman2009search}, where lots of problems in software
engineering field, like software
testing suites \cite{ali2010systematic}, can be solved
by using searching algorithms. Researchers also are aware of those ``magic'' parameters associated with searching algorithms and several studies were conducted to investigate their impacts on the performance of the algorithm in SBSE. Arcuri et al. \cite{arcuri2013parameter} studied parameter tunings in context of unit test data 
generation using grid-search and response surface methodology.
Their results show that tuning does indeed have impact on the performance of searching algorithms. Parterson et al.\cite{paterson2015parameter} performed parameter control to improve performance in unit
test generation problem with three existing methods from {\it Deterministic parameter control}, {\it Adaptive parameter control}, and {\it Self-adaptive parameter control}. Sayyad et al. \cite{sayyad2013parameter} carried out a replicated study on tuning
parameters on Indicator-Based Evolutionary Algorithm (IBEA), and Nondominated Sorting Genetic Algorithm (NSGA-II) using sort of grid-search method. Their results confirm the findings in the original study by Arcuri.

In  machine learning field,  learning algorithms also involves setting parameters associated with the algorithms. For example, in Random Forests\cite{breiman2001random}, there are {\it hyper-parameters} called {\it number of estimators}, {\it depth of the tree} and {\it number of features}, etc. The actual ``good'' learning algorithm used in practice is usually the one after carefully setting those hyper-parameters in order to minimize the generalization error or cost function\cite{larsen1998adaptive,chapelle2002choosing,cherkassky2004practical,bengio2000gradient}. Bengio \cite{bengio2000gradient} proposed to use gradient-based methods to optimize hyper-parameters in linear regression and time-series prediction. It was also proposed by Larsen et al.\cite{larsen1998adaptive} to optimize the regularization parameters in a neural network. To optimize SVM parameters. Chapelle et al. proposed a framework to use gradient descent method\cite{chapelle2002choosing} and Cherkassky investigated analytical methods based on training data and estimated noise level\cite{cherkassky2004practical}. Nevertheless, as pointed by Bergstra \cite{Bergstra2012}, for some learning algorithms, like random forests and CART, we can't evaluate the expectation over the unknown natural distribution of the cost function or even don't have an explicit cost function as linear regression or SVM. Instead using gradient-based method, direct search methods are more feasible for these learning algorithms. Grid search with cross-validation method is an easy way to find optimal hyper-parameters \cite{lerman1980fitting,hsu2003practical,jimenez2009finding,akay2009support}.However, Bergstra \cite{Bergstra2012} demonstrate that grid search is inferior to random search because for most data sets only a few of the tuning parameters really matter, grid search with  sufficient granularity is not efficient for hyper-parameter optimization for most data sets. Therefore, more efficient searching algorithms are needed.

\subsection{Motivation}

 Software analytics is the analytics on data collected from software development
 processes for software managers and engineers to gain insights of their projects
 to make better decisions and deliver high quality software~\cite{menzies2013software}.
 Thanks to the development in Internet and software engineering, there's huge amount
 of data generated from software projects. For example, at the time of this writing (Feb 2016),
 our web searches show that Mozilla Firefox has over 1.1 million bug reports, and
 platforms such as GitHub host over 14 million projects. The PROMISE repository
 of SE data has grown to 200+ projects~\cite{promise15} and this is just one of
 over a dozen open-source repositories that are readily available to researchers~\cite{rod12}.
 The useful and insightful information is always hidden behind the project data,
 which is impossible to obtain by investigate the raw data manually by software
 researchers and practitioners due to the volume of the data. Therefore, some
 automatic and intelligent techniques are required to assist software analytics tasks.
 
 Over the past decade, data mining algorithms are well
 recognized tools to explore software data to learn useful patterns or predictive
 models. Several tasks and practices within software development have been
 improved by applying results of software analytics, from requirement, implementation, testing and maintenance.
%  For example, in software development process, bug
%  reports provide crucial information to developers to fix bugs and improve the
%  software quality. The issues with bug reports that many software teams faced
%  are how to eliminate duplicated reports, how to classify bug reports according
%  to priority, whether the bug reports provide sufficient information to reproduce,
%  and how to triage bug reports and find the right software engineers to fix the bug etc. 
%  Usually these tasks can be done by human beings spending a lot of time, but like
%  1.1 million bug reports in Mozilla Firefox, automation is very important and
%  necessary in this scenario. 
By using text data in software development, data mining algorithms like SVM, linear regression, Naive Bayes, K-nearest neighbors, Decision tree and other algorithms have shown great powerful in addressing problems like
duplicate bug reports detection \cite{sun2010discriminative,jalbert2008automated,alipour2013contextual,nguyen2012duplicate}, bug reports classification \cite{antoniol2008bug,zanetti2013categorizing,lamkanfi2011comparing,tian2013drone}, measure bug report quality \cite{bettenburg2008makes} and triage bug reports \cite{anvik2006should, bhattacharya2010fine, lin2009empirical}. By using software development process data,
researchers could estimate project
efforts\cite{dejaeger2012data,kocaguneli2012value, menzies2013local}, predict risks of 
development delay \cite{da2014empirical,choetkiertikul2015predicting}, and predict defective
module within projects\cite{me07b,hall11,lessmann2008benchmarking}; By mining security
data within software context, software practitioners could predict identifiers of software vulnerabilities\cite{shin2011evaluating,scandariato2012predicting,medeiros2014automatic}.
Similar to investigations on bug reports, all these software analytics 
are exploring data by utilizing data mining algorithms like SVM, Naive Bayes, and
Random Forests\cite{lessmann2008benchmarking, me07b, choetkiertikul2015predicting} to build 
predictive models. In all of these software analytics works,  most of these authors claim
their tools achieve pretty good performance or even better than the baseline methods in terms
of some evaluation measures, like accuracy, F1-measure, and precision \cite{lamkanfi2011comparing, alipour2013contextual,lin2009empirical,lessmann2008benchmarking,hall11}.

Although data mining algorithms are core tools and extensively used in software analytics, software engineering 
researchers rarely develop their own versions of algorithms from scratch. Instead, there are
several open source software tools available, among which most popular ones are Weka (Java)\cite{hall2009weka}, 
Scikit-learn(Python)\cite{scikit-learn}, and numerous data mining packages in R. Thanks to those open 
source tools, they alleviate the burden of coding up those classic data mining algorithms and make
software researchers' life easy. However, this sort of convenience might put 
software analytics at risk because researchers tend to put lots of efforts on problem formalization, 
data collection and data processing and then simply use data mining tools as a black box to build 
predictive models without any considerations regarding the tool.\cite{sun2010discriminative,jalbert2008automated, antoniol2008bug,zanetti2013categorizing,lamkanfi2011comparing,tian2013drone, alipour2013contextual,lin2009empirical,hall11,me07b,choetkiertikul2015predicting,anvik2006should, bhattacharya2010fine}.

% \bi
% \item {\em -I}: the number of trees (default: 100).
% \item {\em -K}: the number of features (default: unlimited).
% \item {\em depth}: the maximum depth of the trees (default: unlimited).
% \item {\em -B}: break ties randomly when several attributes look equally good.
% \item {\em -S}: the seed for random number generator (default:1).
% \ei

% In Scikit-learn, we have the following parameters to control Random Forests behavior\footnote{http://goo.gl/Lfi3IH}:

% \bi
% \item {\em n\_estimators}: the number of trees (default: 10)
% \item {\em max\_features}: the number of features (default: $\sqrt{n\_features}$, options: log2(n\_feautres), etc).
% \item {\em max\_depth}: the maximum depth of the trees (default: nodes are expanded all leaves are pure or until all leaves contain less than min\_samples\_split samples).
% \item {\em min\_samples\_split}: the minimum number of samples required to split an node (default: 2).
% \item {\em min\_samples\_leaf}: the minimum number of samples in newly created leaves (default:1)
% \item {\em random\_state}: the seed for random number generator(default:numpy.random).
% \ei

% In randomForest R package, we have the following parameters \footnote{https://goo.gl/ht4pZ1}:
% \bi
% \item {\em ntree}:  the number of trees to grow (default:500).
% \item {\em mtry}: the number of features randomly sampled as candidates at each split (default: ${\sqrt{n\_features}}$).
% \item {\em nodesize}: minimum size of terminal nodes (default: 5).
% \item {\em maxnodes}: maximum number of terminal nodes trees in the forest can have(default: unlimited).
% \ei

% By looking at the tuning parameters for these three different versions of Random Forest, 
One of the ``black arts'' of data mining is setting the tuning
parameters that control the choices within a data miner. Different implementations might
provide different tuning parameters. Take Random Forests classifiers for example. By comparing the implementation in Weka\footnote{http://goo.gl/hqol4H}, 
Scikit-learn\footnote{http://goo.gl/Lfi3IH} and R package\footnote{https://goo.gl/ht4pZ1}, on one hand, we observe that they have different sets of built-in tuning parameters to control the algorithms. That means software analytics results obtained by
using one tool might not be easily reproduced by using the other one.
On the other hand, even thought they all have similar parameters to control the same behavior like number of trees,
the default values vary a lot from 10 to 500. Therefore, we could believe that those parameters would change
the the behavior of a data miner, to some degree. Nevertheless, we rarely tuned our predictors in
any software analytics tasks since we reasoned that a data miner's default tunings have been 
well-explored by the developers of those algorithms (in which case
tuning would not lead to large performance improvements). Also, we suspected that
tuning would take so long time and be so CPU intensive that the benefits gained would not be worth effort.

There are  a few work within software analytics studying 
the impacts of parameters in learnining algorithm. Corazza et al.\cite{corazza2010effective} explore hyper-parameter space of SVR by tabu-searh and find that parameter tuning
actually improves the performance of effort estimator. However, tabu-search might not be suited to the optimisation of models with continuous variables and of higher dimensionality (e.g. having five or more independent management options to investigate) \cite{mayer1998tabu}. Lessmann et al.\cite{lessmann2008benchmarking} tuned parameters for some of their defect predictors using  a {\em grid search}. Similarly, Tantithamthavorn et at.\cite{tantithamthavorn2016automated} applied {\em caret} package \cite{kuhn2008caret}, which is a grid-search based package,  to defect predictors. By reproducing Tantithamthavorn's results, grid serach
takes extremely long time to find optimal parameters\wei{put running time here}. As mentioned before, grid search explore much searching space which
are not actually useful. This gives
 us a hint that  more simple and efficient tuning
 frameworks should be explored.

To investigate the effect of parameter tuning on software analytics, we take defect prediction as
an example to study how tuning would change conclusions claimed by previous researchers. Since defect prediction
has been explored so well during the past years, tremendous works within this field make it as a very 
active field in software analytics. Furthermore, defect prediction share many similarities with other
software analytics tasks, like delay prediction, effort estimation and bug report classification,
in terms of utilizing data miners as its core tool. Therefore, in this paper, we study effect of tuning
on defect prediction.

\subsection{Research Questions}
By a case study of 17 groups of defect prediction data sets from open source java system,  the research questions addressed in this paper are the following:

\bi
\item RQ1: {\bf{\em Does   tuning    improve the performance scores of a predictor?}} %We will show below
%  examples of truly dramatic improvement:
%  usually by 5 to 20\% and often by much more (in one extreme case,  precision improved from 0\% to 60\%).
\item RQ2: {\bf {\em Does tuning change conclusions on what learners are better than others?}} 
% Recent SE papers~\cite{lessmann2008benchmarking,hall11} claim that some learners are better than others. 
% Some of those conclusions are completely changed by tuning. 
\item RQ3: {\bf {\em Does tuning change conclusions about what factors are most important in software engineering?}} %Numerous recent SE papers (e.g.~\cite{bell2013limited,rahman2013how,me02k,Moser:2008,zimmermann2007predicting,%
% herzig2013predicting}) use data miners to conclude that {\em this}
% is more important than {\em that} for reducing software project defects.
% Given the  tuning results of this paper, we show that such conclusions need to be revisited.
\item  RQ4: {\bf {\em Is tuning easy?}} %We show that one of the simpler multi-objective optimizers
% (differential evolution~\cite{storn1997differential}) works very well for tuning defect predictors. 
\item RQ5: {\bf {\em Is tuning impractically slow?}} %We achieved dramatic improvements in the performance scores
% of our data miners in less than 100 evaluations (!); i.e., very
% quickly.
\item RQ6: {\bf{\em Should data miners be used ``off-the-shelf'' with their default tunings?}}
% For defect prediction from static code measures, our answer is an emphatic ``no'' (and
% the implication for other kinds of  analytics is now an open and urgent question).
\ei

\subsection{Contribution of this thesis}

Based on our answers to above research questions,  we strongly advise that:
\bi
\item
Data miners should not be used ``off-the-shelf'' with default tunings.
\item
Any future paper on defect prediction should include a 
tuning study. Here, we have found  an algorithm called differential
evolution to be a useful method for conducting such
tunings.
\item
Tuning needs to be repeated
whenever data or goals are changed.
Fortunately, the cost of finding good tunings is not excessive since, at least for
static code defect predictors, tuning is easy and fast.
\ei

\subsection{Statement of thesis}

The results of this paper show that, at least for
defect prediction from  code attributes, parameter tuning for defect predictors is {\em remarkably simple} and can {\em dramatically improve the performance}. 

\subsection{Publications (submitted) to date}

\bi
\item W. Fu, T. Menzies, X. Shen ``Tuning for Software Analytics: is it Really Necessary?'', Information and Software Technology, Elsevier, submitted.

\item JC. Nam, W. Fu, S. Kim, T. Menzies, L. Tan ``Heterogeneous Defect Prediction'', Transactions on software engineering, IEEE, submitted.
\ei

\subsection{Structure of this thesis}

The remainder of this paper is organized as follows. Section 2 reviews the literature on defect prediction, tuning in defect prediction , tuning algorithms and associated problems in recent work. Section 3 presents the data sets and  experiment design of our study . Section 4 discusses the results of our experiment. Section 5 discloses the reliability and validity of our study. Finally, section 6 draws the conclusion. Finally, section 7 discusses the next step and future work.
% Faced with this data overload,
% researchers in empirical SE
% use  data miners  to generate 
% {\em defect predictors from static code measures}.
% Such   measures can be
% automatically extracted from the code base, with very little effort
% even for very large software systems~\cite{nagappan05}. 



%Before beginning, we digress for one  clarification.
%This paper is {\em not} arguing that
%software analytics is somehow wrong-headed or misguided.
%In the age of the Internet and global access to software engineering data,
%there exists the  problem of information overload. {\em Something} must be %done to
%allow analysts to make conclusions via an automatic analysis over a lot of %data.
%The results of this paper is that for a particular local context
%(a specific data set and a specific goal) there exists  
%methods for optimizing the conclusions reached in that context.  
%Those conclusions
%may not generalize to other contexts but this  is not a council for despair. 
%As shown here, 
%there  exists general methods for finding
%local conclusions in a particular context. Further,
%those
%methods are  very simple to implement and very fast to execute.

\section{Literature Review} 


\subsection{Defect Prediction}


This section discusses defect prediction,
which is the particular
task explored by our optimizers.
Note that this section repeats much of 
our standard introduction to defect prediction~\cite{me15:book1},
as well as presenting    some new results from Rahman et al.~\cite{rahman14:icse}. 
 



\input{dm101}

 


\subsection{Tuning: Important and Ignored}

In  other fields, the impact of tuning is well understood~\cite{Bergstra2012}. 
Yet issues of tuning  are rarely or poorly addressed
in the defect prediction literature.
When we tune a data miner, what we are really doing is changing how a learner applies
its heuristics. This means tuned data miners use different heuristics, which means they ignore different possible models, which means they return different models; i.e.  {\em how} we learn changes {\em what} we learn.

Are the impacts of tuning addressed in the defect prediction literature?
To answer that question,  in Feb 2016 we searched scholar.google.com for the conjunction of ``data mining" and ``software engineering" and  ``defect prediction"\footnote{More details can be found at https://goo.gl/Inl9nF}.
After sorting by the citation count and discarding the non-SE papers (and those without a pdf link), we read over this sample
of  50 highly-cited SE defect prediction papers. 
What we found in that sample was that few authors
acknowledged the impact of tunings (exceptions:~\cite{Gao:2011,lessmann2008benchmarking}).
Overall,  80\% of papers in our sample {\em did not} adjust
the ``off-the-shelf'' configuration of the data miner (e.g.~\cite{me07b,Moser:2008,Elish2008649}). Of the remaining papers:
\bi
\item
Some papers in our sample  explored   data super-sampling~\cite{4271036} or data sub-sampling techniques via  automatic methods (e.g. ~\cite{Gao:2011,me07b,4271036,Kim:2011}) 
or via some domain principles (e.g. ~\cite{Moser:2008,Nagappan:2008,Hassan:2009}).
As an example of the latter, Nagappan et al.~\cite{Nagappan:2008} checked if metrics related to organizational structure were relatively more powerful for predicting software defects. 
However, it should be noted that  these studies varied the input data but
not the   ``off-the-shelf''   settings of the data miner.
\item
A few other papers did acknowledge that one data miner may not be appropriate
for all data sets.  Those papers tested  different  
``off-the-shelf'' data miners on the same data set.
For example, Elish et al.\cite{Elish2008649}  compared support vector
machines to other data miners for the purposes of defect prediction. SVM's execute via a ``kernel function'' which should be specially selected for different data sets and
the Elish et al. paper  makes no mention of any SVM tuning study.  
To be fair to Elish et al., we hasten to add that we
ourselves have  published
papers using ``off-the-shelf'' tunings~\cite{me07b} since,
prior to this paper it was unclear to us how to effectively
navigate the large space of possible tunings.
\ei
Over our entire sample, there was only  one paper that conducted a somewhat extensive tuning study.
Lessmann et al.\cite{lessmann2008benchmarking} tuned parameters for some of their algorithms using  a {\em grid search}; i.e. divide all $C$ configuration
options into $N$ values, then try all   $N^C$ combinations.
This is a slow approach-- we have explored grid search for 
defect prediction and found it takes days to terminate~\cite{me07b}.
Not only that, we found that grid search can miss
important optimizations~\cite{baker07}.
Every grid has ``gaps'' between each grid division which means
that a supposedly rigorous grid search can still miss
important configurations~\cite{Bergstra2012}. 
Bergstra and Bengio~\cite{Bergstra2012} comment that for most data sets only a few of the tuning parameters really matter-- which means that
much of the runtimes associated with grid search is actually wasted.
Worse still, Bergstra and Bengio  comment that 
the 
important tunings are   different   for different
data sets-- a 
 phenomenon makes grid search a poor choice for configuring data mining
 algorithms for new data sets. 
 



%We found only one paper in our sample from scholar.google.com 
%%that explored anything like the range of configuation values
%that we explore in this paper. Lessmann et %al.~\cite{lessmann2008benchmarking
%}
%explreod 22 learners
%Worse, in the rare paper
%that explores those options, it does so using a method that
%is known to perform poorly (the {\em grid search} method, discussed %below~\cite{Bergstra2012}).
 




%%Further, several  prominent IEEE TSE papers~\cite{lessmann2008benchmarking,hall11,me07b} have claimed 
%that learnerX is better than learnerY for some software analytics task.
%For example, a recent IEEE TSE article claimed that the 
%CART decision tree learner was far worse than Random Forests for
%software defect prediction~\cite{lessmann2008benchmarking}. 
%Such conclusions do not survive tuning.
%For example,
%after tuning, the worst learner (CART) can perform just as well as the supposedly
%best learner (Random Forest). Hence, all those prior results that ranked learners for software
%analytics now need to be revisited (and perhaps revised).

%Returning now to the issue
%of using  ``off-the-shelf'' tunings  for data mining tools: previously
%we have defended that approach, arguing that it
%encourages reproducibility~\cite{me15:book1}. Based on the results
%of this paper, it must be said that that ``off-the-shelf''  policy
%can no longer be condoned. For example, suppose our default
%number of trees in  a Random Forest was   $F=100$ (which is the default in our implementation).
%After tuning, we see that our optimizer often uses $F$ values that are nowhere near that default:

% {\scriptsize
% \[F \in \left\{\begin{array}{l} 55,  65, 70,   82, 88, 96, 100,  102,  104, 107,\\
%                                 108,  119, 133,  140, 140,   147,  145,  142   \end{array}\right\}
% \]}


% {\scriptsize
% \[F \in \left\{\begin{array}{l} 50, 59, 63, 63, 69, 71, 81, 85,85, 88,\\
%                                 92, 95, 96, 99, 99,  100, 104, 106, 107,\\
%                                 109, 111, 112, 121, 121, 123, 125, 128,\\
%                                 131, 132, 135, 140, 141, 141, 150   \end{array}\right\}
% \]}




Since the Lessmann et al. paper, much progress has been made in 
configuration algorithms
and we can now report that  {\em finding useful tunings is very easy}.
This result is both novel and unexpected.
A standard run of grid search (and other  evolutionary algorithms)
is  that optimization requires   thousands,
if not millions, of evaluations.  However, in a result that we found startling, that  {\em differential evolution} (described below) can find useful settings for learners generating defect predictors
in less than 100 evaluations (i.e. very quickly).
Hence,   the ``problem'' (that
tuning changes the conclusions) is really
an exciting opportunity. At least for defect prediction,
 learners are very   amenable to tuning. Hence, 
 they are  also very
amenable to significant performance improvements. Given the low
number of evaluations required, then we assert that tuning
  should be standard practice
for anyone building defect predictors.

%That said, the bad news is that, for defect
%prediction, tuning dramatically changes
%  what is learned from that data. Therefore, it is now an open and %pressing research issue to check if
%analytics without parameter tuning is considered {\em harmful} or, at %the 
%very least, {\em misleading}.
%Clearly, we  must now revisit all prior results   based on
%``off-the-shelf'' tunings.
%Further,
%it is no longer enough to just run a data miner and report the result
%{\em without} first conducting a tuning optimization study.

% \section{Motivating Example}\label{sect:eg}

% This section is  a  demonstration
% of the impact of tuning.
 
% Suppose  a researcher wants to use linear regression
% to test if Halstead's~\cite{halstead77} measures
% of   function complexity
% (number of symbols programmers has to understand) are   {\em better than}
% mere lines of code for predicting
% software defects.  That researcher might believe that Halstead's cognitive approach to
% software bugs is better suited to code refactoring tools since it offers 
% more ways to alter functions that just some coarse grain lines of code measure.


% To test that belief, our  researcher applies regression to  some defect logs.
%  Here are two equations (learned from the NASA data at goo.gl/pGDfvp)
% that use just lines of code or the Halstead measures $N,V,L,D,I,E,B,T$ seen in a
% software module (in this case, a  function).
% Note that the Halstead correlation $c_2$
% is worse than the $c_1$ correlation  from lines of code-- which suggests
%  our researcher should not use Halstead .

% {\scriptsize \[
% \begin{array}{l|l|ll}
% \mathit{measures} & d= \mathit{\#defects} & \mathit{correlation}\\\hline
% \mathit{LOC}   &d_1= 0.0164 +0.0114\mathit{LOC}\ & c_1 = 0.65\\\hline
% \mathit{Halstead} & d_2= 0.231 + 0.00344N  +  0.0009V    \\    
%                  &   - 0.185L- 0.0343D      - 0.00541I  \\ 
%                  & + 0.000019E + 0.711B  - 0.00047T  & c_2=-0.36  
% \end{array}
% \]
% }
 

% \noindent
% We now explore how tuning can change the above  conclusion. Suppose the  predictors $d_1$ and $d_2$  learned from LOC or Halstead
% are used to call an inspection
% team to check for errors in   parts of the code using:
% \begin{equation}\label{eq:yesno}\scriptsize
% \mathit{inspect}= \left\{
% \begin{array}{ll}
% d_i \ge T \rightarrow \mathit{Yes}\\
% d_i <   T \rightarrow \mathit{No} 
% \end{array}\right.
% \end{equation}
% \fig{pd1} shows the effects of tuning. Not surprisingly,
% at $T=0$, all modules get inspected so the false alarm rate is very high. To reduce that
% problem, we can increase $T$:   the false alarm rate falls below
% 20\% at $T=0.45$ (for Halstead). 

 

% \begin{figure}[!t] 

% \renewcommand{\baselinestretch}{0.8}
% {\scriptsize
% \begin{center}

% \% recall (probability of detection):   

% \includegraphics[width=3in]{lsrvscostpd.pdf}

% \% false alarms:

% \includegraphics[width=3in]{lsrvscostpf.pdf}
% \end{center}}
% \caption{
%  Y-axis shows probability of false alarm,
%   and
%   probability of recognizing defective modules  seen using \mbox{$d_i \ge T$}.
%   Curves calculated from the KC2 dataset from the PROMISE repository goo.gl/pGDfvp.
%  }\label{fig:pd1}
%  \end{figure}
 
% Note that either the Halstead or LOC detector can reach some desired
% level of recall, regardless of their correlations, just by
% selecting the appropriate threshold value. For example, in \fig{pd1}, see the recall=75\% values
% found at {\em either} $d_i\ge 0.65$ or $d_2\ge 0.45$ (and at the threshold, the false alarm rates
% were very similar: 14\% and 19\%).

% The  point here is that   the true value of a detector
% could not be assessed {\em without} conducting a  tuning study in the context of some business case (in this case, 
% issuing a request to an inspection team to review some module).  
% Hence, it is important to explore tuning.

% The rest of this paper repeats the analysis of this section, but for 
% more complex learners.
 
%  \subsection{You Can't Always Get What You Want}\label{sect:goals}
 
%  Having made the case that tuning needs to be explored more,
%  but before we get into the technical details of this
%  paper, this section discusses some
%  general matters about setting goals during tuning
%  experiments.
 
%  This paper characterizes tuning as an optimization problem (how to change the settings on the learner
%  in order to best improve the output).
% With such optimizations,  it is not always possible to optimize for all goals at the same time.
% For example, the following text does not
% show results for tuning on recall
% or false alarms since optimizing {\em only} for those goals can lead
% to some undesirable side effects:
% \bi
% \item
% {\em Recall} reports the percentage of  predictions that are actual examples of  what we are looking for.
% When we tune for {\em recall}, we can achieve near
% 100\% recall-- but the cost of a near 100\% false alarms.
% \item
% {\em False alarms} is the percentage of other examples that are reported  (by the learner)
% to be part of the targeted examples.
% When we tune for {\em false alarms}, we 
% can achieve near zero percent false alarm rates by effectively turning off
% the detector (so the  recall falls to nearly zero).
% \ei
% Accordingly,  this paper  explores performance measures that comment on all 
% target classes: see the  precision and ``F'' measures discussed below: see {\em Optimization Goals}.
% That said, we are sometimes asked what good is a learner if it optimizers for (say) precision
% at the expense of (say) recall. 

% Our reply is that software engineering is a very diverse enterprise
% and that different kinds of development need to optimize for different goals
% (which may not necessarily be ``optimize for recall''):
% \bi
% \item
% Anda, Sjoberg and Mockus are concerned with {\em reproducibility}  and so
% assess their models using the the ``coefficient of variation'' $C\frac{stddev}{mean}$) ~\cite{anda09}.
% \item
% Arisholm~\&~Briand~\cite{arisholm06},  Ostrand \& Weyeuker~\cite{ostrand04} and Rahman et al.~\cite{rahman12} are concerned with reducing the work load associated with someone
% else reading a learned model, then applying it. Hence, they assess their models using  {\em reward}; i.e.   the fewest lines of code
%   containing the most bugs.
% \item
% Yin et al. are concerned about
%  {\em incorrect bug fixes}; i.e. those that require subsequent work in order to complete the bug fix.
% These bugs occur  when (say) developers try to fix parts of the code
% where they have very little experience~\cite{yin11}.  Hence, they assess a learned
% model using a measure that selects for  the most number of bugs in regions that {\em the most programmers have worked with before}.
% \item
% For safety critical applications, high false alarm rates are  acceptable if the cost
% of overlooking  critical issues outweighs the inconvenience of   inspecting a few more
% modules. 
% \item
% When rushing a product to market,  there is a business case to 
% avoid the extra rework associated with false alarms.  In that business context, 
% managers might be willing to lower the recall somewhat in order to minimize the false alarms.
% \item
% When the second author worked with contractors at  NASA's software independent verification
% and validation facility, he found  new contractors  
% only reported issues that were most certainly important defects; i.e. they minimized
%   false alarms even if that damaged their precision (since, they felt, 
% it was better to silent than wrong). Later on, once
% those contractors had acquired a reputation of being insightful members of the team,
% they improved their precision scores (even if it means some more false alarms).
% \ei
% Accordingly, this paper does not assume that (e.g.) minimizing false alarms is 
% more important than maximizing   precision or recall. Such a determination 
% depends on   business conditions.

% Rather, what we can  show  examples where  changing  optimization goals can also change 
% the conclusions made from that learner on that data. More generally, we caution that it is 
% important not to overstate  empirical results from  analytics.
% Those results need to be expressed {\em along with} the context within which they are
% relevant (and by ``context'', we mean the optimization goal).



\subsection{Tuning Algorithms}
  
 \wei{will add more details talking about tuning algorithms.here on Monday.}  
How should researchers select which optimizers to apply to tuning data miners?
Cohen~\cite{cohen95} advises comparing new 
 methods against the simplest possible alternative. 
Similarly, Holte~\cite{holte93} recommends using very simple  learners
 as a
kind of ``scout'' for a  preliminary analysis of a data
set (to check if that data really requires a more
complex analysis).
Accordingly,
to find our ``scout'',  we used engineering judgement to sort  candidate algorithms from simplest to  complex. For
example, here is a list of optimizers used widely in research:
{\em 
simulated annealing}~\cite{fea02a,me07f};
 various {\em genetic algorithms}~\cite{goldberg79} augmented by
techniques such as {\em differential evolution}~\cite{storn1997differential}, 
{\em tabu search} and {\em scatter search}~\cite{Glover1986563,Beausoleil2006426,Molina05sspmo:a,4455350};
{\em particle swarm optimization}~\cite{pan08}; 
numerous {\em decomposition} approaches that use
    heuristics to decompose the total space into   small problems,   then apply a
    {\em response surface methods}~\cite{krall15,Zuluaga:13}.
Of these,  the simplest are simulated annealing (SA)  and 
differential evolution (DE), each of which can be coded in less than a page of some high-level scripting language. Our reading of the current literature is that there are more  advocates for
differential evolution than
  SA. For example,  Vesterstrom and Thomsen~\cite{Vesterstrom04} found DE to be competitive with 
   particle swarm optimization and other GAs. 
   
DEs have been applied before for   parameter tuning (e.g. see~\cite{omran2005differential, chiha2012tuning}) but this is the first time they have been applied to
optimize defect prediction from static code attributes.  
The pseudocode for differential evolution is shown in Algorithm~\ref{alg:DE}.
In the following description, 
    superscript numbers denote lines in that pseudocode.

\input{algo} 

%  \begin{table*}[!t]

% \renewcommand{\baselinestretch}{0.8}
% \scriptsize
% \centering
%   \begin{tabular}{c c c c c c c c c c }\hline
%   Dataset &antV0&antV1&antV2&camelV0&camelV1&ivy&jeditV0&jeditV1&jeditV2
% \\\hline
%   training &20/125 &40/178 &32/293 &13/339 &216/608 &63/111 &90/272 &75/306 &79/312
% \\  tuning  &40/178 &32/293 &92/351 &216/608 &145/872 &16/241 &75/306 &79/312 &48/367
% \\  testing &32/293 &92/351 &166/745 &145/872 &188/965 &40/352 &79/312 &48/367 &11/492
% \\ \hline
%   Dataset &log4j&lucene&poiV0&poiV1&synapse&velocity&xercesV0&xercesV1
% \\\hline
%   training &34/135 &91/195 &141/237 &37/314 &16/157 &147/196 &77/162 &71/440
% \\  tuning  &37/109 &144/247 &37/314 &248/385 &60/222 &142/214 &71/440 &69/453
% \\  testing &189/205 &203/340 &248/385 &281/442 &86/256 &78/229 &69/453 &437/588
% \\  \end{tabular}

%   \caption{Data used in this experiment. 
%   E.g., the top left data set has 20 defective classes out of 125 total.
%   See \tion{dataa} for explanation of {\em training, tuning, testing} sets.
%   }\label{tab:data1}
% \end{table*} 

 \begin{table*}[!t]

\renewcommand{\baselinestretch}{0.8}
\scriptsize
\centering
%   \begin{tabular}{p{0.75cm}p{0.75cm}p{0.75cm}p{0.75cm}p{0.75cm}p{0.75cm}p{0.75cm}p{0.75cm}p{0.75cm}p{0.75cm}}\hline
  \begin{tabular}{c c c c c c c c c c } \hline
  Dataset &antV0&antV1&antV2&camelV0&camelV1&ivy&jeditV0&jeditV1&jeditV2
\\\hline
  training &20/125 &40/178 &32/293 &13/339 &216/608 &63/111 &90/272 &75/306 &79/312
\\  tuning  &40/178 &32/293 &92/351 &216/608 &145/872 &16/241 &75/306 &79/312 &48/367
\\  testing &32/293 &92/351 &166/745 &145/872 &188/965 &40/352 &79/312 &48/367 &11/492
\\ \hline
  Dataset &log4j&lucene&poiV0&poiV1&synapse&velocity&xercesV0&xercesV1
\\\hline
  training &34/135 &91/195 &141/237 &37/314 &16/157 &147/196 &77/162 &71/440
\\  tuning  &37/109 &144/247 &37/314 &248/385 &60/222 &142/214 &71/440 &69/453
\\  testing &189/205 &203/340 &248/385 &281/442 &86/256 &78/229 &69/453 &437/588
\\  \end{tabular}

   \caption{Data used in this experiment. 
   E.g., the top left data set has 20 defective classes out of 125 total.
   See \tion{dataa} for explanation of {\em training, tuning, testing} sets.
   }\label{tab:data1}
\end{table*} 

DE evolves a {\em NewGeneration} of candidates  from
a current {\em Population}.  Our DE's lose one ``life''
when the new population is no better than  current one (terminating when ``life'' is zero)$^{L4}$.
Each candidate solution in the {\em Population}  
is a pair of {\em (Tunings, Scores)}.  {\em Tunings} are selected from
\tab{parameters} and {\em Scores} come from training a learner using those parameters
and applying it     test data$^{L23-L27}$.

The premise of DE  is that the best way to mutate the existing tunings
is to {\em Extrapolate}$^{L28}$
between current solutions.  Three solutions $a,b,c$ are selected at random.
For each tuning parameter $i$, at some probability {\em cr}, we replace
the old tuning $x_i$ with $y_i$. For booleans, we use $y_i= \neg x_i$ (see line 36). For numerics, $y_i = a_i+f \times (b_i - c_i)$   where $f$ is a parameter
controlling  cross-over.  The {\em trim} function$^{L38}$ limits the new
value to the legal range min..max of that parameter.
 
The main loop of DE$^{L6}$ runs over the {\em Population}, replacing old items
with new {\em Candidate}s (if  new candidate is better).
This means that, as the loop progresses, the {\em Population} is full of increasingly
more valuable solutions. This, in turn, also improves  the candidates, which are {\em Extrapolate}d
from the {\em Population}.

For the experiments of this paper, we collect performance
values from a data mining, from which a {\em Goal} function extracts one 
performance value$^{L26}$ (so we run this code many times, each time with
a different {\em Goal}$^{L1}$).  Technically, this makes a  {\em single objective} DE (and for notes on multi-objective DEs, see~\cite{Coello05,zhang07,5583335}).


%\begin{algorithm}
%\begin{algorithmic}[1]
% \KwData{this text}
% \KwResult{how to write algorithm with \LaTeX2e }
% initialization\;
% \While{not at end of this document}{
%  read current\;
%  \eIf{understand}{
%   go to next section\;
%   current section becomes this one\;
%   }{
%   go back to the beginning of current section\;
%  }
% }
% \caption{How to write algorithms}
% \end{algorithmic}
%\end{algorithm}





\section{Experimental Design}\label{sect:design}
 



\subsection{Data Sets}\label{sect:dataa}

Our defect data comes from the PROMISE repository \footnote{http://openscience.us/repo}
and pertains to 
open source Java systems defined in terms of \tab{ck}:  {\it ant}, {\it camel}, {\it ivy}, {\it jedit}, {\it log4j}, {\it lucene},
{\it poi}, {\it synapse}, {\it velocity} and {\it xerces}. 

An important principle in data mining is not to test on the data used
in training.  There are many ways to design a experiment that satisfies this principle.
Some of those methods have  limitations; e.g.
{\em leave-one-out} is too slow for large data sets and
{\em cross-validation} mixes up older and newer data  (such that
data from the {\em past} may be used to test on {\em future data}).

To avoid these problems, we used an incremental learning approach. The following
experiment ensures that the training data was created at some time before the test
data.
For this experiment, we use data sets with at least three  
consecutive releases  (where release $i+1$ was built after release $i$). When tuning a learner,
\bi 
% \item {\em Tuned learner}: the {\em first} release is used  on line 24 of Algorithm~\ref{alg:DE} to
%   build some model using some the tunings found in some {\em Candidate}. The {\em second} release was used on line 25 of Algorithm~\ref{alg:DE} to 
%   test the model found on line 24. The {\em third} release was used as a testing data to gather the performance statistics
%   reported below from the best model found by DE.
% \item {\em untuned learner}:



\item The {\em first} release was used  on line 24 of Algorithm~\ref{alg:DE} to
   build some model using some the tunings found in some {\em Candidate}.
 \item The {\em second} release was used on line 25 of Algorithm~\ref{alg:DE} to 
   test the candidate model found on line 24.
   \item Finally the {\em third} release was used to gather the performance statistics
   reported below from the best model found by DE.
 \ei
 
 To be fair for the untuned learner, the {\em first} and {\em second} releases used in tuning experiments
 will be combined as the training data to build a model. Then the performance of this untuned learner 
 will be evaluated by the same {\em third} release as in the tuning experiment.
 
% This approach ensures   all treatments 
% are assessed on the same tests. Note that we did consider
% one other experimental design but rejected it for reasons of
% internal validity (see \tion{construct}).

Some data sets have more than three releases and, for those data, we could run more
 than one experiment. For example, {\em ant} has five versions in PROMISE so
 we ran three experiments called V0,V1,V2:
 \bi
 \item AntV0: first, second, third = versions 1, 2, 3
 \item AntV1: first, second, third = versions 2, 3, 4
 \item AntV2: first, second, third = versions 3, 4, 5
 \ei 
These data sets are displayed in \tab{data1}.

% As an aside, an alternate experimental design would be to 
% learn a baseline learner from the first {\em and} second release
% instead of, as shown above,  just the first release. On the one hand,
% this would mean that the baseline could be learned from more data.
% On the other hand, this adds a conflation to our experimental design
% since the optimizer uses the second release for pruning, not growing a data set.  Happily, from 
% piror work~\cite{Menzies:2008aa} we know that defect predictors usually {\em saturates} (i.e.
% does not generate better predictors) after 100 examples, which is a number smaller than all our first release data sets. Hence, their would
% be little value in generating the baselines using the first and second
% releases. 

 \subsection{Data Miners}
 
There are several ways to make defect predictors
using  CART~\cite{brieman00}, Random Forest~\cite{breiman84}, 
 WHERE~\cite{menzies2013local} and LR (logistic regression).
For this study, we use CART, Random Forest and LR versions  from 
SciKitLearn~\cite{scikit-learn} and
WHERE, which is available from
github.com/ai-se/where. 
 We use  these algorithms for the following reasons.
 
CART and Random Forest were mentioned in
a recent IEEE TSE paper by Lessmann et al.~\cite{lessmann2008benchmarking} that compared 22  
learners for  defect prediction.
That study ranked  CART  worst  and Random Forest as best.
In a demonstration of the impact of tuning,
this paper shows  we can {\em refute} the conclusions of  Lessmann et al.
in the sense that, after tuning,
CART
performs just as well as
 Random Forest.

LR was  mentioned by Hall et al.~\cite{hall11}
as usually being as good or better as more complex learners (e.g.
Random Forest). In a finding that endorses the Hall et al. result,
we show that untuned LR performs better than 
untuned Random Forest (at least, for the data sets studied here). However,
we will show that tuning raises doubts about the optimality of the
Hall et al. recommendation.

Finally,  this
 paper uses WHERE since, as shown below,
it offers the new perspective that tuning could change the
most important features selected by the learner.
  
%%%%%%%%%%%%%%%% list of parameters%%%%%%%%%%%%%%%%%%%%%
\renewcommand\arraystretch{1.2}
\begin{table*}[t!]
\scriptsize
  \centering
	\begin{tabular}{|c|c|c|c|l|}
	\cline{1-5}
	\begin{tabular}[c]{@{}c@{}}Learner Name\end{tabular} & Parameters & Default &\begin{tabular}[c]{@{}c@{}}Tuning\\ Range\end{tabular}& 
\multicolumn{1}{c|}{Description} \\ \hline
 	\multirow{8}{*}{\begin{tabular}[c]{@{}c@{}}Where-based\\ Learner\end{tabular}}
%  	WHERE-based  Learner} 
	& threshold & 0.5 &[0.01,1]& The value to determine defective or not .\\ \cline{2-5} 
	& infoPrune & 0.33 &[0.01,1]& The percentage of features to consider for the best 
split to build its final decision tree. \\ \cline{2-5} 
	 & min\_sample\_split & 4& [1,10]& The minimum number of samples required to split an internal node of
its final  decision tree. \\ \cline{2-5} 
	 & min\_Size & 0.5 &[0.01,1]& \begin{tabular}[c]{@{}l@{}}Finds min\_samples\_leaf 
in the initial clustering tree using ${n\_samples}^ {min\_Size}$.
\end{tabular} \\ \cline{2-5} 
    & wriggle & 0.2 &[0.01, 1] & The threshold to determine which branch in  the initial clustering tree to be pruned\\ \cline{2-5}
	 & depthMin & 2 & [1,6]&The minimum depth of the initial clustering tree below which no pruning for the
clustering tree. \\ \cline{2-5} 
	 & depthMax & 10 &[1,20]& The maximum depth of the initial clustering tree. \\ \cline{2-5} 
	 & wherePrune & False &T/F& Whether or not to prune the initial clustering tree. \\ \cline{2-5}
	 & treePrune & True &T/F& Whether or not to prune the final decision tree. \\ \cline{2-5} 
\hline
\multirow{5}{*}{CART} & threshold & 0.5 &[0,1]& The value to determine defective or not. \\ \cline{2-5} 
	 & max\_feature & None &[0.01,1]& The number of features to consider when looking for the best 
split. \\ \cline{2-5} 

	 & min\_sample\_split & 2 &[2,20]& The minimum number of samples required to split an 
internal node. \\ \cline{2-5} 
	 & min\_samples\_leaf & 1 & [1,20]&The minimum number of samples required to be at a leaf 
node. \\ \cline{2-5} 
     & max\_depth & None & [1, 50]& The maximum depth of the tree. \\
     \cline{1-5}  
       \multirow{5}{*}{\begin{tabular}[c]{@{}c@{}}Random \\ Forests\end{tabular}}  & threshold & 0.5 & [0.01,1] & The value to determine defective or not. \\ 
\cline{2-5} 
	 & max\_feature & None &[0.01,1]& The number of features to consider when looking for the best 
split. \\ \cline{2-5} 
	 & max\_leaf\_nodes & None &[1,50]& Grow trees with max\_leaf\_nodes in best-first fashion. \\ \cline{2-5} 
	 & min\_sample\_split & 2 &[2,20]& The minimum number of samples required to split an 
internal node. \\ \cline{2-5} 
	 & min\_samples\_leaf & 1 &[1,20]&The minimum number of samples required to be at a leaf 
node. \\ \cline{2-5} 
	 &  n\_estimators & 100 & [50,150]&The number of trees in the forest.\\ \cline{2-5}
	 \hline 
Logistic Regression&\multicolumn{4}{c|}{This study uses untuned LR in order to check
a conclusion of~\cite{hall11}. }\\\hline

	\end{tabular}
    \caption {List of parameters tuned by this paper.}
\label{tab:parameters}
\end{table*}

\subsection{ Learners' Tunings}


Our learners use the tuning parameters of \tab{parameters}. This section describes those parameters.
The default parameters for CART and Random Forest are set by 
the SciKitLearn authors and the
default parameters for WHERE-based learner are set via our own expert judgement.
When we say a learner is used ``off-the-shelf'', we mean
that they use the defaults shown in \tab{parameters}. 

As to the value of those defaults, it could be argued that these defaults are 
not the best  parameters needed for practical defect prediction.
That said,  prior to this paper, two things were true:
\bi
\item 
Many data scientists in SE use the standard defaults
in their data miners, 
without   tuning (e.g.~\cite{me07b,Moser:2008,herzig2013predicting,zimmermann2007predicting}).
\item
The effort involved to adjust those tunings seemed so onerous, that
many researchers in this field were content to take our prior advice
of ``do not tune... it is just too hard''~\cite{me15:book1}.
\ei
As to why we used the "Tuning Range" shown in \tab{parameters}, and not some other ranges,
we note that (1)~those ranges included the defaults; (2)~the results shown below
show that by exploring those ranges,   we achieved large gains in the performance of our defect predictors.
This is not to say that {\em larger} tuning ranges might not result in {\em greater} improvements.
However, for the goals of this paper (to show that some tunings do matter), exploring
just these ranges shown in \tab{parameters} will suffice.


% As to the details of these learners, LR is a parametric
% modeling approach. Given $f = \beta_0 + \sum_i\beta_ix_i$,
% where $x_i$ is some measurement in a data set, and $\beta_i$
% is learned via regression, LR
% converts that into a function $0 \le g \le 1$
% using $g=1/\left(1+e^{- f}\right)$. This function reports how much
% we believe in a particular class. 

CART, Random Forest, and WHERE-based learners are all  tree learners that divide a data set, then recur
on each split.
All these learners
generate numeric predictions which are converted
into binary ``yes/no'' decisions via \eq{yesno}.

\begin{equation}\label{eq:yesno}\scriptsize
\mathit{inspect}= \left\{
\begin{array}{ll}
d_i \ge T \rightarrow \mathit{Yes}\\
d_i <   T \rightarrow \mathit{No} ,
\end{array}\right.
\end{equation}
The splitting process is controlled by numerous tuning parameters.
If data contains more than {\em min\_sample\_split}, then a split is attempted.
On the other hand, if a split contains no more than {\em min\_samples\_leaf}, then the recursion stops. CART and Random Forest use a 
user-supplied constant for this parameter while
WHERE-based learner firstly computes this parameter $m$={\em min\_samples\_leaf} from the size of the data
sets via  $m=\mathit{size}^\mathit{min\_size}$ to build an initial  clustering tree.
Note that WHERE builds {\em two} trees: the initial clustering tree (to find similar sets of data)
then a final decision tree (to learn rules that predict for each similar cluster)\footnote{A
frequently asked question is why does WHERE build two trees--
would not   a single tree suffice? The answer is, as shown below,  tuned WHERE's twin-tree approach 
generates very precise predictors.}.
The tuning parameter  {\em min\_sample\_ split } controls the construction of the
final decision tree (so, for WHERE-based learner,
{\em min\_size} and {\em min\_sample\_split} are the parameters to be tuned).

These learners use different techniques to explore the splits:
\bi
\item
CART finds the attributes whose ranges contain rows with least variance in the number
of defects\footnote{If an attribute ranges $r_i$ is found in 
$n_i$ rows each with a  defect count variance of $v_i$, then CART seeks the attributes
whose ranges minimizes $\sum_i \left(\sqrt{v_i}\times n_i/(\sum_i n_i)\right)$.}.
\item
Random Forest    divides data like CART then  builds $F>1$  trees,
each time using some random subset of
the attributes. 
\item
When building the initial cluster tree, WHERE projects the data on to a dimension it synthesizes from the raw data using
a process analogous to principle component analysis\footnote{
PCA  synthesises  new
attributes $e_i, e_2,...$
that extends across the dimension of greatest  variance in the data  with attributes $d$.  
This process  combines
redundant  variables into a smaller set of variables  (so $e \ll d$) since those
redundancies become (approximately) parallel lines
in $e$ space. For all such redundancies \mbox{$i,j \in d$}, we 
can ignore $j$ 
since effects that change over $j$ also
change in the same way over $i$.
PCA is also useful for skipping over noisy variables from $d$-- these
variables are effectively ignored since    they  do not contribute to the variance in the data.}.
WHERE  divides  at the median point of that projection.
On recursion,
this generates the initial clustering tree, the leaves of which are clusters of  very similar examples. After that, when building 
the final decision tree, WHERE pretends its clusters are ``classes'', then 
asks the InfoGain of the
Fayyad-Irani discretizer~\cite{FayIra93Multi}, to rank the attriubutes, where {\em infoPrune} is used.
WHERE's final decision tree generator then ignores everything except the top   {\em infoPrune} percent of the sorted
attributes.
\ei
Some tuning parameters are learner specific:
\bi
\item
{\em Max\_feature} is used by
CART and Random Forest to select the number of attributes
used to build one tree.
CART's default is to use all the attributes while 
Random Forest usually selects the square root of the number
of attributes.
\item
  {\em Max\_leaf\_nodes} is the upper bound on leaf notes generated in a 
  Random Forest.
\item {\em Max\_depth} is the upper bound on the depth of the CART tree.  
 \item
WHERE's  
tree generation will always split up to {\em depthMin} number of branches.
After that, WHERE will only split data if the mean performance scores of the two halves
is ``trivially small'' (where ``trivially small'' is set by the   {\em wriggle} parameter). 
\item
WHERE's   {\em tree\_prune} setting controls how   
WHERE prunes back superflous parts of the final decision tree. 
If a decision sub-tree and its parent have the same 
majority cluster
(one that occurs most frequently), then if {\em tree\_prune} is enabled, we prune that decision sub-tree.
\ei


\subsection{Optimization Goals}

Recall from Algorithm~1 that we call differential evolution once for each
 optimization goal. This section lists those optimization goals.
Let $\{A,B,C,D\}$ denote the
true negatives, 
false negatives, 
false positives, and 
true positives
(respectively) found by a binary detector. 
Certain standard measures can be computed from
$A,B,C,D$, as shown below. Note that for $pf$, the {\em better} scores are {\em smaller}
while
for all other scores, the {\em better} scores are {\em larger}.

{\scriptsize\[
\begin{array}{ll}
pd=recall=&D/(B+D)\\
pf=&C/(A+C)\\ 
prec=precision=&D/(D+C) \\
F =&2*pd*prec/(pd + prec)
\end{array}
\]}

The rest of this paper explores tuning for {\em prec} and {\em F}. Our point is not that these are best or most important optimization goals.
Indeed, the list of ``most important'' goals is domain-specific
and we only explore these two to illustrate how conclusions can change dramatically
when moving from one goal to another.

\section{Experimental Results}

In the following, we explore the effects of tuning WHERE, Random Forest,
and CART. LR will be used, untuned, in order to check one of the recommendations
made by Hall et al.~\cite{hall11}.

\subsection{RQ1:  Does  Tuning  Improve Performance? }\label{sect:precision}


\fig{deltas} says  that the answer to RQ1 is ``yes''-- tuning  has a positive effect on performance scores. This figure sorts
 deltas in the precision and the F-measure    between tuned and untuned learners. Our reading of this
figure is that, overall, tuning rarely makes performance   worse and often can make it much better. 
 

\begin{figure}[!t]
\begin{center}
\includegraphics[width=1.5in]{./eps/improvements_precision.eps}\includegraphics[width=1.5in]{./eps/improvements_F.eps}
 \end{center}
\caption{Deltas in performance  seen in \tab{precisionbars} (left)
and \tab{fbars} (right) between tuned and untuned learners. Tuning improves performance when the deltas are above zero.}\label{fig:deltas}
 \end{figure}
 
 
\tab{precisionbars} and \tab{fbars} show the
the specific values seen before and after tuning with {\em precision}
and {\em ``F''} as different optimization goals(the corresponding  ``F'' and precision values for
\tab{precisionbars} and \tab{fbars} are not provided for the space limitation).
For each data set, the maximum precision or ``F'' values for each data set are shown in {\bf bold}.
As might have been
 predicted by Lessmann et al.~\cite{lessmann2008benchmarking}, 
untuned CART is indeed the worst learner (only one of its
untuned results is best and {\bf bold}). 
And, 
in $\frac{12}{17}$ cases, the  untuned Random Forest performs better than or equal to untuned CART in terms of precision.  
% Note that the size of the improvement is sometimes small; e.g. the precision results improve more than the ``F'' measure.
% But even when the {\em median} change is small, there still may exist
% interesting (i.e.  exceptionally
% large) improvements from tuning-- for example, 
%  see the last  two ``F'' improvements for WHERE in \fig{deltas} where the improvements were greater than 70\%.
% Some of the data sets in \fig{precisionbars} proved challenging for all learners;
% e.g. the precision results for {\em ivy} are less
% that impressive.
% To some extent, this is due to
% the properties of   the data set (as shown in \fig{data1}, defective classes in {\em ivy} are very rare in tuning data).




That said,  tuning can improve those poor performing detectors.
In some cases, the median changes may be small (e.g. the ``F'' results for WHERE and Random Forests) but even in
those cases, there are enough large changes to motivate the use of tuning. For example:
\bi
\item
For ``F'' improvement, there are two improvements over 25\% for both WHERE and Random Forests. Also, in {\em poiV0}, all untuned learners report ``F'' of under 50\%, tuning changes those scores by 25\%. Finally, note the  {\em xercesV1} result for the WHERE learner. Here, tuning changes precision from 32\% to 70\%.
\item
Regarding precision, for {\em antV0}, and {\em antV1} untuned WHERE reports precision of 0. But tuned WHERE scores 35 and 60 (the similar pattern can seen in ``F'').

\ei

% To complete the discussion in this section, we note that in the Lessmann et al. study,  Random Forest dramatically out-performed CART.
% In this study, we show that  
% tuned CART is now comparable to Random Forest. So our new results
% do not complete reverse the results of Lessmann et al. However, they
% do show that the Lessmann results are ``brittle'', in the sense that
% tuning can remove the effect they report.

\begin{table}[!t]
\renewcommand{\baselinestretch}{0.8} 
% \centering
\scriptsize    

\begin{tabular}{r|rl|rl|rl|rl|rl|rlrl}
%\begin{tabular}{r@{~}|r@{~}l@{~}|r@{~}l@{~}|r@{~}l|r@{~}l@{~}|r@{~}l@{~}|r@{~}l@{~}r@{~}l}
      &   \multicolumn{4}{c|}{WHERE}         &   \multicolumn{4}{c|}{CART}         &   \multicolumn{4}{c}{Random Forest}         \\\hline
  Data set   &   \multicolumn{2}{c}{default}         &   \multicolumn{2}{c|}{Tuned}         &   \multicolumn{2}{c}{default}         &   \multicolumn{2}{c|}{Tuned}    &   \multicolumn{2}{c}{default}  &   \multicolumn{2}{c}{Tuned}\\\hline
antV0 & 0 &   & 35 &   & 15 &   & {\bf 60} &   & 21 &   & 44 &  \\
antV1 & 0 &   & 60 &   & 54 &   & 56 &   & {\bf 67} &   & 50 &  \\
antV2 & 45 &   & 55 &   & 42 &   & 52 &   & 56 &   & {\bf 67} &  \\
camelV0 & 20 &   & 30 &   & 30 &   & 50 &   & 28 &   & {\bf 79} &  \\
camelV1 & 27 &   & 28 &   & {\bf 38} &   & 28 &   & 34 &   & 27 &  \\
ivy & 25 &   & 21 &   & 21 &   & {\bf 26} &   & 23 &   & 20 &  \\
jeditV0 & 34 &   & 37 &   & 56 &   & {\bf 78} &   & 52 &   & 60 &  \\
jeditV1 & 30 &   & 42 &   & 32 &   & {\bf 64} &   & 32 &   & 37 &  \\
jeditV2 & 4 &   & {\bf 22} &   & 6 &   & 17 &   & 4 &   & 6 &  \\
log4j & 96 &   & 91 &   & 95 &   & 98 &   & 95 &   & {\bf 100} &  \\
lucene & 61 &   & 75 &   & 67 &   & 70 &   & 63 &   & {\bf 77} &  \\
poiV0 & 70 &   & 70 &   & 65 &   & {\bf 71} &   &  67 &   & 69 &  \\
poiV1 & 74 &   & 76 &   & 72 &   & 90 &   & 78 &   & {\bf 100} &  \\
synapse & 61 &   & 50 &   & 50 &   & {\bf 100} &   & 60 &   & 60 &  \\
velocity & 34 &   & {\bf 44} &   & 39 &   & {\bf 44} &   & 40 &   & 42 &  \\
xercesV0 & 14 &   & 17 &   & 17 &   & 14 &   & {\bf 28} &   & 14 &  \\
xercesV1 & 86 &   & 54 &   & 72 &   & {\bf 100} &   & 78 &   & 27 &  \\
\end{tabular}
\caption{Precision results (best results  shown in {\bf bold}).}
\label{tab:precisionbars}
\end{table}
 

\begin{table}[!t]
\renewcommand{\baselinestretch}{0.8} 
% \centering
\scriptsize  
~~~\begin{tabular}{r|rl|rl|rl|rl|rl|rlrl}
      &   \multicolumn{4}{c|}{WHERE}         &   \multicolumn{4}{c|}{CART}         &   \multicolumn{4}{c}{Random Forest}         \\\hline
  Data set   &   \multicolumn{2}{c}{default}         &   \multicolumn{2}{c|}{Tuned}         &   \multicolumn{2}{c}{default}         &   \multicolumn{2}{c|}{Tuned}    &   \multicolumn{2}{c}{default}  &   \multicolumn{2}{c}{Tuned}\\\hline
antV0 & 0 &   & 20 &   & 20 &   & {\bf 40} &   & 28 &   & 38 &  \\
antV1 & 0 &   & 38 &   & 37 &   & {\bf 49} &   & 38 &   & {\bf 49} &  \\
antV2 & 47 &   & 50 &   & 45 &   &  49 &   & {\bf 57} &   & 56 &  \\
camelV0 & 31 &   & 28 &   & 39 &   & 28 &   & {\bf 40} &   & 30 &  \\
camelV1 & 34 &   & 34 &   & 38 &   & 32 &   & {\bf 42} &   & 33 &  \\
ivy & 39 &   & 34 &   & 28 &   & {\bf 40} &   & 35 &   &  33 &  \\
jeditV0 & 45 &   & 47 &   & 56 &   & 57 &   & {\bf 63} &   & 59 &  \\
jeditV1 & 43 &   & 44 &   & 44 &   & 47 &   & 46 &   & {\bf 48} &  \\
jeditV2 & 8 &   & {\bf 11} &   & 10 &   & 10 &   & 8 &   & 9 &  \\
log4j & 47 &   & 50 &   & 53 &   & 37 &   & {\bf 60} &   & 47 &  \\
lucene & 73 &   & 73 &   & 65 &   & 72 &   & 70 &   & {\bf 76} &  \\
poiV0 & 50 &   & 74 &   & 31 &   & 64 &   & 45 &   & {\bf 77} &  \\
poiV1 & 75 &   & {\bf 78} &   & 68 &   & 69 &   & 77 &   & {\bf 78} &  \\
synapse & 49 &   & 56 &   & 43 &   & {\bf 60} &   & 52 &   & 53 &  \\
velocity & 51 &   & 53 &   & 53 &   & 51 &   & {\bf 56} &   & 51 &  \\
xercesV0 & 19 &   & 22 &   & 19 &   & {\bf 26} &   & 34 &   & 21 &  \\
xercesV1 & 32 &   & 70 &   & 34 &   & 35 &   & 42 &   & {\bf 71} &  \\
\end{tabular}
\caption{F-measure results (best results  shown in {\bf bold}).}
\label{tab:fbars}
\end{table}

\subsection{RQ2:  Does Tuning Change a Learner's Ranking ?}\label{sect:rank}
Researchers often use performance criteria to assert that one learner is better than 
another~\cite{lessmann2008benchmarking,me07b,hall11}. For example:
\be
\item
Lessmann et al.~\cite{lessmann2008benchmarking} conclude that
Random Forest is considered to be statistically 
better than CART. 
\item
Also, in Hall et al.'s   systematic literature review\cite{hall11}, it is argued
that defect predictors based on simple 
modeling techniques such as LR perform better than complicated techniques such as Random Forest\footnote{By three measures,
Random Forest
is more complicated than LR. Firstly, LR builds one model
while Random Forest builds many models. Secondly, LR is just
a model construction tool while Random Forest needs both
a tool to construct its forest {\em and} a second tool
to  infer some conclusion from all the members of that forest.
Thirdly, the LR model can be printed in a few lines while the multiple
models learned by Random
Forest model would take up multiple pages of output.}.
\ee
Given tuning, how stable are these  conclusions?
Before answering issue, we digress for two comments.



Firstly, it is important to comment on why it is  so important to check the conclusions
of these particular papers. 
These  papers are prominent publications (to say the least).
Hall et al.~\cite{hall11} is the fourth most-cited IEEE TSE
paper for 2009 to 2014 with 176 citations (see goo.gl/MGrGr7)
while the Lessmann et al. paper~\cite{lessmann2008benchmarking} has 394 citations (see
goo.gl/khTp97)-- which is quite remarkable for a paper published in 2009.
Given the prominence
of these papers, researchers might believe it is
appropriate to
  use  their advice without testing that advice on local data sets.

Secondly, while we are critical of the results of
Lessmann et al. and Hall et al., it needs to be said that  their analysis  was 
excellent and exemplary given the state-of-the-art of the tools used when those papers were written.  
While Hall et al. did not perform any new experiments, 
their
summarization of so many defect prediction papers have not been equalled
before (or since).
As to the Lessmann et al. paper, they  compared
22 data miners using various   data sets (mostly from NASA)~\cite{lessmann2008benchmarking}.
In that study, some learners were tuned using manual methods 
(C4.5, CART and Random Forest)
and some, like SVM-Type learners, were tuned by automatic grid search (for more on grid search, see   {\S}2.1).



\begin{figure}[!t]
% \begin{center}
\includegraphics[width=1.5in]{./eps/LR_untuned.eps}\includegraphics[width=1.5in]{./eps/LR_tuned.eps}
%  \end{center}
\caption{Comparison between Logistic Regression and Random Forest before and after tuning. }\label{fig:lr}
 \end{figure}
 
 
That said, our tuning results show that it is time to revise
the recommendations of those papers. 
  \fig{lr} comments on the advice from Hall et al. (that LR is better than Random Forest)L
  \bi
  \item
 In a result that might have been predicted by Hall et al.,  
 untuned Random
Forests performs comparatively worse than
 Logistic Regression. Specifically, untuned
 Random Forest performs worse than Linear regression in in 13 out of   17 data sets. 
\item
However, it turns out that advice is sensitive to the tunings
used with Random Forest. After tuning, we find that tuned Random Forest
loses to Logistic Regression in only 6 out of 17 data sets. 
\ei

As to Lessmann et al.'s advice (that Random Forest is better than 
CART),  
in \tab{precisionbars} and \tab{fbars}, we saw those counter-examples
to that statement.
Recall in those tables,
tuned CART are better than or equal
to tuned Random Forest in $\frac{12}{17}$ and $\frac{7}{17}$ data sets in
terms of precision and F-measure, respectively. Prior to tuning experiments, those
numbers are $\frac{5}{17}$ and $\frac{1}{17}$. Results from the non-parametric
Kolmogorov-Smirnov(KS) Test show that the performance 
scores of tuned CART and tuned Random Forest are not statistically different.
Note that Random Forest is  not significantly better than CART, would not have been
predicted by   Lessmann et al.
  

Hence we answer RQ2 as ``yes'': tuning can change how  data miners are comparatively ranked.
 


% \begin{figure}[!t]
% \begin{center}
% \includegraphics[width=1.5in]{svm_rbf.eps}\includegraphics[width=1.5in]{svm_sigmoid.eps}
%  \end{center}
% \caption{Comparison between tuned and naive SVM learners with rbf and sigmoid kernels over the goal of F. }\label{fig:svm}
%  \end{figure}








 \subsection{RQ3: Does Tuning Select Different Project Factors? }\label{sect:import}


Researchers often use data miners to  test what factors have most impact on software projects~\cite{bell2013limited,rahman2013how,me02k,Moser:2008,zimmermann2007predicting,herzig2013predicting}. 
\tab{features} comments that such tests are unreliable since the factors selected by a data miner are much altered before and 
after tuning.

\tab{features} shows what features are found in the trees generated by the WHERE algorithm
(bold shows the features found by the trees from tuned WHERE; plain text shows the features seen
in the untuned study). Note that different features are selected depending on whether or not
we tune an algorithm.


% list some features
\begin{table}[!t]

\renewcommand{\baselinestretch}{0.8}
\scriptsize
\centering
  \begin{tabular}{c|p{1in}|p{1in}}
    \multicolumn{1}{c|}{ Data set}  &   \multicolumn{1}{c|}{Precision} & \multicolumn{1}{c}{F} \\ \hline 
 \multirow{2}{*}{antV0} & {\bf rfc} &  {\bf None} \\
         & mfa, loc, cam, dit, dam, lcom3 & mfa, loc, cam, dit, dam, lcom3\\
  \hline
 \multirow{2}{*}{camelV0} & {\bf mfa, wmc, lcom3} &{\bf None }\\
        & mfa, wmc, rfc, loc, cam, lcom3 & mfa, wmc, rfc, loc, cam, lcom3\\
  \hline
 \multirow{2}{*}{ivy} & {\bf cam, dam, npm, loc, rfc, wmc} &{\bf cam, dam, npm, loc, rfc, wmc }  \\
       & loc, cam, dam, wmc, lcom3 & loc, cam, dam, wmc, lcom3 \\
  \hline
 \multirow{2}{*}{jeditV0} &{\bf mfa, dam, loc }&{\bf mfa, dam, loc}\\
         & mfa, lcom3, dam, dit, ic & mfa, lcom3, dam, dit, ic \\
  \hline
 \multirow{2}{*}{log4j} & {\bf loc, ic, dit }&{\bf mfa, wmc, rfc, loc, npm}\\
         & mfa, lcom3, loc, ic & mfa, lcom3, loc, ic \\
   \hline
  \multirow{2}{*}{lucene} & {\bf dit, cam, wmc, lcom3, dam, rfc, cbm, mfa, ic} & {\bf dit, lcom3, dam, mfa}\\
         & dit, cam, dam, ic & dit, cam, dam, cbm, ic\\
   \hline
   \multirow{2}{*}{poiV0} & {\bf mfa, amc, dam }& {\bf mfa, amc, dam} \\
        & mfa, loc, amc, dam, wmc, lcom & mfa, loc, amc, dam, wmc, lcom\\
   \hline
   \multirow{2}{*}{synapse} &{\bf loc, dit, rfc, cam, wmc, dam, lcom,  mfa,  lcom3} & {\bf dam} \\
        & loc, mfa, cam, lcom, dam, lcom3 & loc, mfa, cam, lcom, dam, lcom3\\
    \hline
   \multirow{2}{*}{velocity} & {\bf dit, wmc, cam, rfc, cbo, moa, dam }& {\bf mfa, dit }\\
        & dit, dam, lcom3, ic, mfa, cbm  & dit, dam, lcom3, ic, mfa\\
    \hline
   \multirow{2}{*}{xercesV0} &{\bf  wmc }&{\bf cam, dam, avg\_cc, loc, wmc,  dit,  mfa, ce, lcom3 }\\
        & wmc, mfa, lcom3, cam, dam &  wmc, mfa, lcom3, cam, dam \\
    \hline
    
  \end{tabular}
  
    \caption{Features selected by tuned WHERE with different goals:
    {\bf bold} features are those found useful by the tuned WHERE.
    Also, features shown in plain text are those found useful by the untuned WHERE.
    }\label{tab:features}
\end{table}


For example, consider {\em mfa} which is the
number of methods inherited by a class plus the number of methods accessible by member methods of the class.
For both goals (precision and ``F'') {\em mfa} is selected for 8 and 5 data sets,
for the untuned and tuned data miner (respectively).
Similar differences are   seen with other attributes.


As to why different tunings select for different features,  recall from {\S}2.1 that tuning changes how data miners
heuristically explore a large space of possible models. As we change how that exploration proceeds,
so we change what features are found by that exploration.

In any case, our answer to RQ3 is ``yes'', tuning changes our
conclusions about what factors are most important in software engineering.
Hence, many old papers    need to be revisited  and perhaps revised~\cite{bell2013limited,rahman2013how,me02k,Moser:2008,zimmermann2007predicting,herzig2013predicting}.  
For example, one of us (Menzies) used data miners
to assert that some factors were more important than others for predicting
successful software reuse~\cite{me02k}. That assertion should now be doubted since Menzies did not conduct a tuning study before reporting what factors the data miners
found were most influential.

\begin{figure}[!t]
% \begin{center}
\includegraphics[width=1.5in]{./eps/np_precision.eps}\includegraphics[width=1.5in]{./eps/np_f.eps}
%  \end{center}
\caption{Deltas in performance between {\em np = 10} and the recommended np's. The recommended np is better when deltas are above zero. {\em np = 90, 50 and 60} are recommended population size for WHERE, CART and Random Forest by Storn.}\label{fig:deltas_np}
 \end{figure}



\subsection{RQ4: Is Tuning Easy?}\label{sect:easy}

In terms of the search space
explored via tuning, optimizing defect prediction from static code
measures is much {\em smaller} than the standard optimization.

To see this,
recall from Algorithm~1 that
DE explores a {\em Population} of size {\em np = 10}. This is a very small population size since
Rainer Storn (one of the inventors of DE) recommends  setting {\em np} to be ten times larger than the number
of attributes being optimized~\cite{storn1997differential}.

From \tab{parameters},
we see that Storn would therefore recommend {\em np} values of
90, 50, 60 for WHERE, CART and Random Forest (respectively). Yet we achieve our results
using a constant {\em np = 10}; i.e. $\frac{10}{90}, \frac{10}{50}, \frac{10}{60}$ of the
recommended search space.

To justify that {\em np = 10} is enough, we did another tuning study, 
where all the settings were the same as before but we set {\em np = 90, np = 50} and {\em np = 60} for WHERE, CART and Random Forest, respectively (i.e. the settings
as recommended by Storn). The tuning performance of learners was evaluated
by precision and ``F'' as before. To compare performance of each learner with different {\em np}'s, we computed the delta in the performance between {\em np = 10} and {\em np} using any of \{90, 50, 60\}.

Those deltas, shown in \fig{deltas_np}, are sorted along the x-axis. In those plots, a zero or negative $y$ value means that {\em np = 10} performs as well or better than  {\em np $\in \{90, 50, 60\}$}. One technical aside: the data set orderings in \fig{deltas_np} on the x-axis are not the same (that is,
if {\em np $>$ 10} was useful for optimizing one data set's precision score, it was not necessary for that data set's F-measure score).

  \fig{deltas_np} shows that
the median improvement is zero; i.e. {\em np = 10} usually does as well as anything else. This observation is
supported by the   KS
  results of \tab{ks}. At a 95\% confidence, the
KS threshold  is $1.36\sqrt{34/(17*17)} = 0.46$, which is greater than  the values in \fig{deltas_np}. That is, no result in  \fig{deltas_np} is significantly different to any other-- which is to say that
there is no evidence that   {\em np = 10} is a poor choice of search space size.

Another measure showing that tuning is easy 
(for static code defect predictors)
is the number of evaluations required to complete optimization
(see next section). That is, we answer RQ4 as ``yes'', tuning is surprisingly easy-- at least
for defect predictors and using DE.


% \begin{table}[!t]
% \centering
% \scriptsize
% \begin{tabular}{|l|c|c|}
% \hline
% Learner & CART & WHERE \\ \hline
% CART    & -    & 0.42  \\ \hline
% R.Forest      & 0.29 & 0.18  \\ \hline
% \end{tabular}
% \begin{tabular}{|l|c|c|}
% \hline
% Learner & CART & WHERE \\ \hline
% CART    & -    & 0.24  \\ \hline
% R.Forest      & 0.24 & 0.29  \\ \hline
% \end{tabular}
% \caption{Kolmogorov-Smirnov Tests for numbers seen in \fig{deltas_np}: Precision(left) and F(right).}
% \end{table}


\begin{table}[!t]
\renewcommand{\baselinestretch}{0.8}
\scriptsize
 \centering
  \begin{tabular}{c|c c|cc}
    &   \multicolumn{2}{c|}{Precision} & \multicolumn{2}{c}{F} \\ \hline 
    Learner & CART  & WHERE & CART & WHERE \\
\hline
    CART & - & 0.41 & - & 0.24 \\
    R. Forest &  0.12 & 0.35 & 0.18 & 0.18 \\
  \end{tabular}
    \caption{Kolmogorov-Smirnov Tests for distributions of  \fig{deltas_np}}\label{tab:ks}
\end{table}

\begin{figure}[!t]
\begin{center}
\includegraphics[width=1.5in]{./eps/features_precision.eps}\includegraphics[width=1.5in]{./eps/features_F.eps}
\end{center}
\caption{Four representative tuning values in WHERE with  precision and F-measure as the tuning goal, respectively.   }\label{fig:features}
 \end{figure}

% \begin{table}[!t]
% \renewcommand{\baselinestretch}{0.8}
% \scriptsize
% \centering
%   \begin{tabular}{c|c |c}
%     &   \multicolumn{1}{c|}{Precision} & \multicolumn{1}{c}{F} \\ \hline 
%     Learner  & np = 10 & recommended np \\
% \hline
%     WHERE & 0.18 & 0.24 \\
%     CART  & 0.12 & 0.12  \\
%     R. Forest &  0.12 & 0.12 \\
%   \end{tabular}
%     \caption{Kolmogorov-Smirnov Tests for distributions of each learner with np = 10 and recommended np's }\label{tab:ks}
% \end{table}

% %%% repalce this table with the new one
% \begin{table}[!ht]

% \renewcommand{\baselinestretch}{0.8}
% \scriptsize
% \centering
%   \begin{tabular}{c|c c|c c|c c|c c| c c }
  
%     &   \multicolumn{2}{c|}{Precision} & \multicolumn{2}{c|}{F} &  \multicolumn{2}{c|}{SUM}\\
%  &&&&&&&\\
% Features&   
%   default
% & tuned
% & default
% & tuned
% & default
% & tuned
% \\\hline

% max\_cc& & & &  & & \\
% noc& & & & & & \\
% ca& & & & & & \\
% cbo& & 1& & & & 1\\
% moa& & 1& & & & 1\\
% ce& & 2& & 1& & 3\\
% avg\_cc& & 2& & 2& & 4\\
% npm& 1& 2& 1& 1&2 & 3\\
% lcom& 1& 2& 1& 1& 2& 3\\
% amc& 4& 2& 4& 2& 8& 4\\
% cbm& 5& 2& 5& 3& 10& 5\\
% rfc& 3& 6& 3& 8& 6& 14\\
% wmc& 5& 4& 5& 9& 10& 13\\
% dit& 8& 3& 7& 7& 15& 10\\
% ic& 8& 3& 8& 6& 16 & 9\\
% lcom3& 8& 6& 8& 8&16 & 14\\
% cam& 9& 7& 9& 7& 18& 14\\
% loc& 9& 5& 9& 11& 18& 16\\
% dam& 13& 6& 13& 11& 26& 17 \\
% mfa& 16& 9& 16& 9&32 & 18\\

 
%   \end{tabular}
%     \caption{Counts of features selected by different goals.Given that we are processing 17 data sets, the maximum counts for any 
% one cell in the ``precision'' or ``F'' column is 17.  
%     }\label{tab:counts}
% \end{table}
 

% \begin{table*}[!ht]

% \renewcommand{\baselinestretch}{0.75}
% \scriptsize
% \centering
%   \begin{tabular}{r|c |c |c |c |c |c }
%     Datasets & Tuned\_Where & Default\_Where & Tuned\_CART & Default\_CART & Tuned\_RanFst & Default\_RanFst\\
%     \hline
%     antV0 & 50 / 90.18 & 1.57 & 60 / 4.36 & 0.09 & 50 / 8.12 & 0.18\\
%     antV1 & 50 / 174.67 & 2.90 & 50 / 6.35 & 0.10 & 50 / 11.77 & 0.27\\
%     antV2 & 50 / 403.63 & 6.92 & 70 / 9.71 & 0.16 & 60 / 13.28 & 0.35\\
%     camelV0 & 50 / 537.53 & 8.60 & 50 / 9.14 & 0.18 & 50 / 13.73 & 0.31\\
%     camelV1 & 60 / 1640.54 & 24.57 & 60 / 17.06 & 0.24 & 70 / 31.53 & 0.73\\
%     ivy & 70 / 77.75 & 1.02 & 60 / 3.86 & 0.06 & 50 / 8.00 & 0.17\\
%     jeditV0 & 80 / 472.57 & 5.49 & 60 / 6.30 & 0.09 & 70 / 13.01 & 0.30\\
%     jeditV1 & 60 / 489.45 & 6.82 & 70 / 7.61 & 0.11 & 60 / 12.87 & 0.32\\
%     jeditV2 & 50 / 435.43 & 7.21 & 50 / 6.46 & 0.12 & 90 / 20.34 & 0.37\\
%     log4j & 70 / 113.73 & 1.36 & 70 / 3.25 & 0.05 & 60 / 7.07 & 0.16\\
%     lucene & 70 / 224.39 & 2.70 & 50 / 4.07 & 0.08 & 50 / 8.87 & 0.26\\
%     poiV0 & 60 / 261.06 & 4.00 & 60 / 6.23 & 0.10 & 50 / 10.57 & 0.29\\
%     poiV1 & 80 / 607.85 & 7.18 & 60 / 7.69 & 0.13 & 50 / 11.39 & 0.29\\
%     synapse & 50 / 116.04 & 1.87 & 60 / 4.07 & 0.05 & 70 / 9.74 & 0.16\\
%     velocity & 60 / 195.27 & 2.75 & 60 / 4.49 & 0.06 & 80 / 12.15 & 0.21\\
%     xercesV0 & 60 / 143.69 & 2.17 & 70 / 7.26 & 0.09 & 60 / 10.28 & 0.23\\
%     xercesV1 & 50 / 794.50 & 13.37 & 50 / 8.24 & 0.15 & 60 / 14.54 & 0.38\\

%   \end{tabular}
%   \caption{Evaluations/runtimes and runtimes for tuned and default learners(in sec), optimizing for  precision.  }\label{tab:etimes}
% \end{table*}
% %%%%time for F %%%%%%
% \begin{table*}[!ht]
% \renewcommand{\baselinestretch}{0.75}
% \scriptsize
% \centering
%   \begin{tabular}{r|c |c |c |c |c |c } 
%     Datasets & Tuned\_Where & Default\_Where & Tuned\_CART & Default\_CART & Tuned\_RanFst & Default\_RanFst\\
%     \hline
%     antV0 & 50 / 94.08 & 1.71 & 60 / 4.55 & 0.08 & 60 / 10.79 & 0.21\\
%     antV1 & 60 / 193.74 & 3.02 & 70 / 7.77 & 0.09 & 60 / 12.30 & 0.25\\
%     antV2 & 80 / 643.94 & 7.59 & 60 / 8.38 & 0.15 & 70 / 16.99 & 0.41\\
%     camelV0 & 60 / 662.56 & 9.97 & 60 / 13.19 & 0.23 & 80 / 26.11 & 0.32\\
%     camelV1 & 60 / 1800.64 & 24.25 & 50 / 15.02 & 0.28 & 50 / 28.52 & 0.78\\
%     ivy & 60 / 69.95 & 1.03 & 50 / 3.35 & 0.08 & 70 / 9.40 & 0.18\\
%     jeditV0 & 90 / 553.80 & 5.58 & 50 / 5.58 & 0.09 & 60 / 15.08 & 0.33\\
%     jeditV1 & 60 / 519.75 & 8.76 & 50 / 7.43 & 0.13 & 60 / 18.13 & 0.41\\
%     jeditV2 & 70 / 621.32 & 8.98 & 50 / 9.71 & 0.15 & 60 / 17.38 & 0.63\\
%     log4j & 70 / 125.29 & 1.73 & 50 / 2.90 & 0.06 & 60 / 8.76 & 0.19\\
%     lucene & 50 / 221.99 & 3.52 & 50 / 5.20 & 0.10 & 50 / 10.09 & 0.33\\
%     poiV0 & 60 / 327.48 & 5.13 & 50 / 6.56 & 0.11 & 50 / 12.88 & 0.36\\
%     poiV1 & 50 / 523.85 & 8.95 & 80 / 12.26 & 0.14 & 60 / 19.56 & 0.35\\
%     synapse & 70 / 148.23 & 1.91 & 60 / 3.96 & 0.06 & 60 / 8.19 & 0.16\\
%     velocity & 50 / 156.51 & 2.75 & 60 / 4.27 & 0.06 & 50 / 7.70 & 0.22\\
%     xercesV0 & 60 / 142.83 & 2.01 & 70 / 7.15 & 0.08 & 60 / 9.61 & 0.20\\
%     xercesV1 & 50 / 751.92 & 12.98 & 60 / 9.28 & 0.16 & 50 / 12.69 & 0.38\\

%   \end{tabular}
%   \caption{Evaluations/runtimes and runtimes for tuned and default learners(in sec), optimizing for F-measure.  }\label{tab:ftimes}
% \end{table*}




%%%%time for prec %%%%%%
\begin{table*}[!ht]
\renewcommand{\baselinestretch}{0.75}
\scriptsize
\centering
  \begin{tabular}{r|c |c |c |c |c |c }
    Datasets & Tuned\_Where & Naive\_Where & Tuned\_CART & Naive\_CART & Tuned\_RanFst & Naive\_RanFst\\
    \hline
    antV0 & 50 / 95.47 & 1.65 & 60 / 5.08 & 0.08 & 60 / 9.78 & 0.20\\
    antV1 & 60 / 224.67 & 3.03 & 50 / 6.52 & 0.12 & 60 / 14.13 & 0.25\\
    antV2 & 70 / 644.99 & 8.24 & 50 / 9.00 & 0.24 & 60 / 16.75 & 0.44\\
    camelV0 & 70 / 690.62 & 7.93 & 70 / 12.68 & 0.24 & 110 / 28.49 & 0.34\\
    camelV1 & 60 / 1596.77 & 23.56 & 60 / 17.13 & 0.27 & 70 / 33.96 & 0.77\\
    ivy & 60 / 66.69 & 0.97 & 60 / 4.26 & 0.07 & 60 / 8.89 & 0.19\\
    jeditV0 & 80 / 459.30 & 5.33 & 80 / 8.69 & 0.11 & 90 / 18.40 & 0.32\\
    jeditV1 & 60 / 421.56 & 6.59 & 80 / 9.05 & 0.12 & 80 / 17.93 & 0.36\\
    jeditV2 & 90 / 595.56 & 6.88 & 60 / 7.90 & 0.14 & 110 / 27.34 & 0.38\\
    log4j & 50 / 76.09 & 1.33 & 50 / 2.60 & 0.06 & 80 / 9.69 & 0.15\\
    lucene & 80 / 236.45 & 2.60 & 70 / 6.07 & 0.10 & 60 / 9.77 & 0.25\\
    poiV0 & 60 / 263.12 & 3.92 & 70 / 7.42 & 0.09 & 130 / 25.86 & 0.28\\
    poiV1 & 50 / 398.33 & 6.94 & 70 / 9.31 & 0.13 & 50 / 12.67 & 0.29\\
    synapse & 70 / 144.09 & 1.85 & 50 / 3.88 & 0.07 & 50 / 8.13 & 0.19\\
    velocity & 60 / 184.10 & 2.68 & 50 / 4.27 & 0.07 & 100 / 15.18 & 0.21\\
    xercesV0 & 60 / 136.87 & 1.98 & 80 / 9.17 & 0.10 & 70 / 14.17 & 0.22\\
    xercesV1 & 80 / 1173.92 & 12.78 & 60 / 10.47 & 0.16 & 50 / 18.27 & 0.40\\
  \end{tabular}
  \caption{Evaluations/runtimes and runtimes for tuned and default learners(in sec), optimizing for precision.}
  \label{tab:etimes}
\end{table*}


%%%%time for F %%%%%%
\begin{table*}[!ht]
\renewcommand{\baselinestretch}{0.75}
\scriptsize
\centering
  \begin{tabular}{r|c |c |c |c |c |c }
    Datasets & Tuned\_Where & Naive\_Where & Tuned\_CART & Naive\_CART & Tuned\_RanFst & Naive\_RanFst\\
    \hline
    antV0 & 50 / 93.38 & 1.39 & 50 / 3.52 & 0.08 & 70 / 9.89 & 0.17\\
    antV1 & 60 / 186.95 & 3.18 & 50 / 6.18 & 0.12 & 60 / 13.39 & 0.25\\
    antV2 & 90 / 654.34 & 8.08 & 60 / 8.79 & 0.18 & 120 / 27.56 & 0.36\\
    camel & 50 / 543.28 & 9.65 & 80 / 17.00 & 0.28 & 70 / 22.52 & 0.41\\
    camelV1 & 60 / 1808.03 & 26.98 & 110 / 31.92 & 0.28 & 70 / 37.00 & 0.85\\
    ivy & 60 / 74.50 & 1.18 & 60 / 4.72 & 0.08 & 60 / 10.39 & 0.21\\
    jeditV0 & 80 / 518.47 & 6.11 & 60 / 7.9
 & 0.10 & 60 / 14.32 & 0.37\\
    jeditV1 & 70 / 576.29 & 6.89 & 70 / 8.13 & 0.10 & 70 / 17.42 & 0.34\\
    jeditV2 & 80 / 657.59 & 7.93 & 70 / 10.34 & 0.15 & 80 / 20.20 & 0.40\\
    log4j & 70 / 123.48 & 1.59 & 50 / 2.92 & 0.08 & 50 / 7.67 & 0.17\\
    lucene & 60 / 219.02 & 3.68 & 60 / 6.89 & 0.12 & 70 / 13.06 & 0.35\\
    poiV0 & 60 / 314.53 & 4.82 & 60 / 7.80 & 0.10 & 80 / 19.29 & 0.32\\
    poiV1 & 50 / 446.05 & 7.55 & 50 / 7.62 & 0.14 & 110 / 27.23 & 0.36\\
    synapse & 60 / 138.75 & 1.83 & 60 / 4.87 & 0.08 & 90 / 13.29 & 0.17\\
    velocity & 60 / 211.88 & 3.13 & 60 / 5.51 & 0.10 & 60 / 11.58 & 0.27\\
    xercesV0 & 80 / 178.49 & 2.02 & 60 / 7.47 & 0.11 & 80 / 17.31 & 0.28\\
    xercesV1 & 80 / 1370.89 & 14.42 & 60 / 11.07 & 0.19 & 80 / 25.27 & 0.46\\
  \end{tabular}
  \caption{Evaluations/runtimes and runtimes for tuned and default learners(in sec), optimizing for F-Measure.}
\label{tab:ftimes}
\end{table*}





\subsection{RQ5: Is Tuning Impractically Slow?}\label{sect:fast}
 
 
The number of evaluations/runtimes used by our optimizers is shown in \tab{etimes}
and \tab{ftimes}.
WHERE's runtimes are slower than CART and Random Forest since WHERE has yet to benefit from decades
of implementation experience with these older algorithms. For example, SciKitLearn's  CART and Random Forest
 make extensive use of an underlying C library whereas WHERE is a purely interpreted Python.

Looking over \tab{etimes} and \tab{ftimes}, the general pattern is that 50 to 80 evaluations suffice for finding the tuning
improvements reported in this paper. 
50 to 80 evaluations are  much fewer than our pre-experimental intuition.
Prior to this paper, the authors have conducted numerous explorations of evolutionary algorithms
for search-based SE applications~\cite{krall15,krall15:hm,fea02a,me07f,Green}. Based
on that work, our expectations were that non-parametric evolutionary optimization would
take thousands, if not millions, of evaluations of candidate tunings. This turned out not
to be that case.

Hence, we answer RQ5 as ``no'': tuning is so fast that
it could (and should) be used by anyone using defect predictors. 
%The possible reason that tuning is so fast is that the searching space for defect prediction is not complicated, which might result from the tuning range set for each parameter in \tab{parameters}.  

As to why DE can tune defect predictors so quickly, that is an
open question. One possibility is that the search space within
the control space of these data miners has  many accumulative effects such that one
decision can cascade into another (and the combination of decisions
is better than each separate  one). DE would be a natural  tool for reasoning
about such ``cascades'', due to the way it mashes candidates together,
then inserts the result back into the frontier (making them available
for even more mashing at the next step of the inference).


\subsection{RQ6: Should we use ``off-the-shelf'' Tunings?}\label{sect:variance}
 
 In \fig{features}, we show how tuning selects the optimal values for tuned parameters. For space limitation, only four parameters from WHERE learner are selected as representatives and all the others can be found in our online support documents\footnote{\url{https://goo.gl/aHQKtU}}.
 Note that
 the tunings learned were different in different data sets and for different goals.
Also, the tunings learned by DE
were often very different to the default (the default values for {\em threshold}, {\em infoPrune}, {\em min\_Size} and {\em wriggle} are $0.5$, $0.33$, $0.5$ and $0.2$, respectively). That is, to achieve the performance improvements seen in the paper,
the default tuning parameters required a wide range of adjustments.

Hence, we answer RQ6 as ``no'' since, to achieve the improvements seen in this paper, tuning has to be repeated whenever the goals or data
sets are changed. Given this requirement to repeatedly run tuning, it is fortunate that (as shown above)
tuning is so easy and so fast (at least for defect predictors from static code attributes).


 
% %%%%parameters for prec %%%%%%
% \begin{table*}[!ht]
 
% \renewcommand{\baselinestretch}{0.9}
% \resizebox{\textwidth}{!}{
% \scriptsize
% \centering
%   \begin{tabular}{|c |c |c |c |c |c |c |c |c |c |c |c |c |c |c |c |c |c |c |c |}
%     \hline
%   \begin{tabular}[c]{@{}c@{}}Learner \\ Name\end{tabular}&Parameters  & Default &antV0&antV1&antV2&camelV0&camelV1&ivy&jeditV0&jeditV1&jeditV2&log4j&lucene&poiV0&poiV1&synapse&velocity&xercesV0&xercesV1\\ 
%  \hline
% \multirow{8}{*}{\begin{tabular}[c]{@{}c@{}}Where\\based\\ Learner\end{tabular}}
% & threshold& 0.5& 0.98& 0.98& 0.43& 0.24& 0.64& 1& 1& 0.98& 0.98& 1& 1& 0.87& 0.59& 0.98& 1& 0.98& 0.98\\ \cline{2-20}
% & infoPrune& 0.33& 0.05& 0.05& 0.71& 0.54& 0.45& 0.41& 0.3& 0.05& 0.05& 0.54& 0.84& 0.01& 1& 0.05& 0.68& 0.43& 0.05\\ \cline{2-20}
% & min\_sample\_size& 4& 7& 7& 9& 8& 6& 10& 1& 5& 7& 8& 7& 9& 3& 7& 7& 1& 7\\ \cline{2-20}
% & min\_Size& 0.5& 0.51& 0.51& 0.59& 0.46& 0.13& 0.38& 0.66& 0.27& 0.51& 0.46& 0.47& 0.77& 0.48& 0.51& 0.66& 0.22& 0.51\\ \cline{2-20}
% & wriggle& 0.2& 0.6& 0.6& 0.83& 0.52& 0.19& 0.01& 0.26& 0.6& 0.6& 0.52& 0.19& 0.83& 0.01& 0.6& 0.26& 0.55& 0.6\\ \cline{2-20}
% & depthMin& 2& 1& 1& 2& 3& 5& 2& 3& 3& 1& 1& 2& 4& 2& 1& 3& 2& 1\\ \cline{2-20}
% & depthMax& 10& 8& 8& 13& 19& 19& 18& 7& 8& 8& 19& 1& 19& 18& 8& 11& 18& 8\\ \cline{2-20}
% & wherePrune& False& False& False& False& False& True& True& False& False& False& False& False& True& True& False& False& True& False\\ \cline{2-20}
% & treePrune& True& False& False& False& True& True& True& False& False& False& True& True& False& False& False& False& True& False\\ \cline{2-20}
% \hline
% \multirow{4}{*}{CART}
% & threshold& 0.5& 0.69& 0.99& 1& 0.3& 0.83& 1& 0.99& 0.58& 0.72& 1& 0.71& 0.46& 0.72& 1& 0.85& 1& 0.64\\ \cline{2-20}
% & max\_feature& None& 0.01& 0.58& 0.65& 0.66& 0.73& 0.67& 0.56& 0.01& 0.97& 0.54& 0.52& 0.32& 0.01& 0.74& 0.73& 0.01& 0.1\\ \cline{2-20}
% & min\_samples\_split& 2& 7& 16& 18& 5& 11& 6& 15& 6& 17& 4& 16& 12& 5& 14& 11& 4& 10\\ \cline{2-20}
% & min\_samples\_leaf& 1& 13& 14& 10& 4& 3& 15& 16& 9& 6& 7& 6& 1& 4& 6& 7& 11& 7\\ \cline{2-20}
% & max\_depth& None& 14& 1& 41& 34& 1& 22& 29& 1& 1& 1& 14& 19& 8& 40& 4& 1& 1\\ \cline{2-20}
% \hline
% \multirow{6}{*}{\begin{tabular}[c]{@{}c@{}}Random \\ Forests\end{tabular}} 
% & threshold& 0.5& 0.84& 0.9& 0.83& 0.33& 1& 0.99& 0.91& 1& 1& 0.83& 0.98& 0.9& 0.86& 0.83& 1& 1& 1\\ \cline{2-20}
% & max\_feature& None& 0.61& 0.13& 0.89& 0.37& 0.01& 0.98& 0.52& 0.75& 0.35& 0.01& 0.98& 0.84& 0.73& 0.01& 0.48& 0.51& 0.01\\ \cline{2-20}
% & max\_leaf\_nodes& None& 37& 35& 38& 21& 36& 45& 10& 38& 10& 30& 20& 43& 11& 13& 15& 39& 10\\ \cline{2-20}
% & min\_samples\_split& 2& 8& 16& 17& 13& 14& 2& 3& 2& 2& 18& 19& 9& 4& 4& 9& 20& 1\\ \cline{2-20}
% & min\_samples\_leaf& 1& 19& 5& 2& 4& 2& 4& 7& 17& 7& 16& 12& 2& 3& 2& 2& 2& 3\\ \cline{2-20}
% & n\_estimators& 100& 138& 112& 77& 74& 125& 130& 107& 85& 96& 111& 103& 82& 59& 149& 150& 63& 58\\ \cline{2-20}
% \hline  \end{tabular}
% }
%   \caption{Parameters tuned on different models over the objective of precision.}\label{tab:preselect}
% \end{table*}


% %%%%parameters for F %%%%%%
% \begin{table*}[!ht]
 
% \resizebox{\textwidth}{!}{
% \renewcommand{\baselinestretch}{0.9}
% \scriptsize
% \centering
%   \begin{tabular}{|c |c |c |c |c |c |c |c |c |c |c |c |c |c |c |c |c |c |c |c |}
%     \hline
    
%   \begin{tabular}[c]{@{}c@{}}Learner \\ Name\end{tabular}&Parameters  & Default &antV0&antV1&antV2&camelV0&camelV1&ivy&jeditV0&jeditV1&jeditV2&log4j&lucene&poiV0&poiV1&synapse&velocity&xercesV0&xercesV1\\ 
%  \hline
% \multirow{8}{*}{\begin{tabular}[c]{@{}c@{}}Where\\based\\ Learner\end{tabular}}
% & threshold& 0.5& 0.04& 0.44& 0.44& 0.98& 0.65& 0.77& 1& 0.65& 0.98& 0.44& 0.44& 0.87& 0.04& 0.77& 0.24& 0.44& 0.77\\ \cline{2-20}
% & infoPrune& 0.33& 0.51& 0.68& 0.88& 0.47& 0.07& 0.31& 0.48& 0.68& 0.57& 0.12& 0.68& 0.01& 0.51& 0.14& 0.54& 0.68& 0.14\\ \cline{2-20}
% & min\_sample\_size& 4& 6& 4& 6& 1& 6& 8& 8& 4& 6& 7& 4& 9& 6& 2& 8& 4& 8\\ \cline{2-20}
% & min\_Size& 0.5& 0.18& 0.4& 0.56& 0.51& 0.65& 0.59& 0.97& 0.4& 0.51& 0.8& 0.4& 0.77& 0.18& 0.62& 0.46& 0.4& 0.66\\ \cline{2-20}
% & wriggle& 0.2& 0.25& 0.29& 0.76& 0.6& 0.63& 0.26& 1& 0.51& 0.17& 0.36& 0.51& 0.83& 0.25& 0.5& 0.52& 0.29& 0.26\\ \cline{2-20}
% & depthMin& 2& 3& 3& 3& 1& 5& 3& 2& 3& 5& 5& 3& 4& 3& 3& 3& 3& 3\\ \cline{2-20}
% & depthMax& 10& 16& 15& 15& 8& 19& 10& 7& 15& 5& 15& 15& 19& 16& 6& 19& 15& 10\\ \cline{2-20}
% & wherePrune& False& False& True& True& True& True& True& True& False& False& True& True& True& False& True& False& False& True\\ \cline{2-20}
% & treePrune& True& False& True& True& False& False& False& False& False& True& True& True& False& False& False& True& True& False\\ \cline{2-20}
% \hline
% \multirow{5}{*}{CART}
% & threshold& 0.5& 0.34& 0.25& 0.01& 0.01& 0.73& 0.53& 0.92& 0.8& 0.74& 0.54& 0.03& 0.91& 0.01& 0.01& 0.55& 1& 0.01\\ \cline{2-20}
% & max\_feature& None& 0.01& 0.01& 0.29& 0.01& 0.46& 0.75& 0.79& 0.74& 0.41& 0.81& 0.61& 0.72& 0.01& 0.01& 0.01& 0.25& 0.18\\ \cline{2-20}
% & min\_samples\_split& 2& 18& 20& 12& 2& 15& 11& 2& 18& 13& 9& 17& 16& 10& 4& 8& 3& 15\\ \cline{2-20}
% & min\_samples\_leaf& 1& 19& 16& 15& 17& 1& 1& 13& 10& 4& 3& 7& 5& 20& 7& 8& 1& 6\\ \cline{2-20}
% & max\_depth& None& 12& 2& 15& 1& 41& 20& 44& 15& 13& 5& 23& 14& 1& 5& 17& 47& 13\\ \cline{2-20}
% \hline
% \multirow{6}{*}{\begin{tabular}[c]{@{}c@{}}Random \\ Forests\end{tabular}} 
% & threshold& 0.5& 0.01& 0.35& 0.3& 0.01& 0.9& 0.97& 0.63& 1& 0.73& 0.68& 0.01& 1.0& 0.01& 0.07& 0.22& 1& 0.82\\ \cline{2-20}
% & max\_feature& None& 0.63& 0.17& 0.01& 0.01& 0.88& 0.74& 0.76& 0.73& 0.01& 0.03& 0.39& 0.02& 0.01& 0.56& 0.36& 0.51& 0.89\\ \cline{2-20}
% & max\_leaf\_nodes& None& 40& 33& 46& 22& 11& 16& 38& 34& 30& 31& 12& 49& 25& 47& 15& 39& 24\\ \cline{2-20}
% & min\_samples\_split& 2& 10& 16& 20& 1& 1& 1& 1& 4& 20& 19& 11& 14& 2& 17& 19& 20& 19\\ \cline{2-20}
% & min\_samples\_leaf& 1& 4& 15& 9& 13& 18& 11& 3& 16& 17& 6& 10& 7& 19& 13& 11& 2& 14\\ \cline{2-20}
% & n\_estimators& 100& 120& 73& 75& 130& 97& 144& 125& 97& 80& 111& 96& 101& 50& 67& 74& 63& 66\\ \cline{2-20}
% \hline  \end{tabular}
% }
%   \caption{Parameters tuned on different models over the objective of ``F''.}\label{tab:fselect}
% \end{table*}



% \section{Guidelines \wei{added!}}

% As discussed above, tuning is helpful and easy to do. Any defect prediction study based on data mining should include a tuning study. However, if tuning is not done properly, performance may not be improved but degraded to some extent. Here're some tips for tuning:

% {\em Tuning Data} is important for the whole tuning process. If available data size for the study is very small, k-fold cross evaluation method can be used to split data and generate new training data as well as tuning data. On the other hand, if the study concerns more about chronology, incremental learning approach can be adopted. 


% {\em Tuning Range} will determine how large the searching space the tuner will explore and how good the parameter got from tuning. The recommended range will set the default value as the median of the range with a reasonable distance. If the performance from tuning dose not better than the default parameter, adjust ranges accordingly.

% {\em Searching algorithm} is the engine of the tuning process. Since tuning has to be repeatedly done for different goals and different data sets, simple searching algorithms, like DE and Generic algorithms, are good choice to complete this task. Other more advanced algorithms, like NSGA III \cite{deb2014evolutionary} and GALE\cite{krall15}, can also be used if multi-objective tuning is considered.


\section{Reliability and Validity}\label{sect:construct}


{\em Reliability} refers to the consistency of the results obtained
from the research.  For example,   how well independent researchers
could reproduce the study? To increase external
reliability, this paper has taken care to either  clearly define our
algorithms or use implementations from the public domain
(SciKitLearn). Also, all the data used in this work is available
on-line in the PROMISE code repository and all our algorithms
are on-line at github.com/ai-se/where.



{\em External validity} checks if the results are of relevance
for other cases, or can be generalized from samples
to populations.  
The examples of this paper  only relate to precision, recall, and the F-measure
but the general principle (that the search bias changes the search conclusions)  holds for any set of goals. 
Also,
the tuning results shown here only came from one  software analytics task 
(defect prediction from static code attributes).
There are many other kinds of software analytics tasks 
(software development effort estimation, social network mining,
detecting duplicate issue reports, etc) and the implication of this
study for those tasks is unclear. 
However,  those other tasks often use the same kinds of learners
explored in this paper so it is quite possible that
the conclusions of this paper apply to other SE analytics tasks as well. 

%That said, there exist some class of data mining papers for which
%tuning may not be required. Consider  Le Goues et al.'s 2012
%ICSE paper that used a evolutionary program to learn
%repairs to code~\cite{leGoues12}. The performance criteria
%in that paper was ``can we fix any of the known bugs?''. Note
%that this criteria is a ``{\em competency}'' statement, and
%not a ``{\em better than}'' statement (the difference being that
%one is 
%``can do'' and the other is ``can do better''). For such
%competency claims, tuning is not necessary. However, as soon
%as {\em better than} enters the performance criteria then this
%becomes a race between competing methods. In such a race,
%it is unfair to hobble one competitor with poor tunings.

\section{Conclusions}


Our exploration of the six research
questions listed in the introduction
show that when learning defect predictors for static code
attributes,   analytics without parameter tuning are considered {\em harmful} and {\em misleading}:
\bi
\item Tuning improves the performance scores of a predictor.
That improvement is usually positive (see \fig{deltas}) and sometimes
it can be quite   dramatic (e.g. precision changing from 0 to 60\%). \item 
Tuning changes conclusions on what learners are better than others.
Hence, it is time to revisit numerous prior publications of our own~\cite{me07b}
and others~\cite{lessmann2008benchmarking,hall11}.
\item
Also,
tuning changes conclusions on what factors are most important in software development.
Once again, this means that old papers may need to be revised including those
some of our own~\cite{me02k} and others~\cite{bell2013limited,rahman2013how,Moser:2008,zimmermann2007predicting,herzig2013predicting}. 
\ei

\section{Future Work}

As to future work, it is now important
to explore and exploit the implications of these
conclusions. In this section, we will discuss potential opportunities and challenges 
based on this paper's results, which will be the future focus for my PhD study.

\subsection{Find a Better Tuner}

This paper has investigated  {\em some} learners using {\em one}  optimizer. Hence, we can make
no claim that DE is the {\em best} optimizer for {\em all} learners.
Rather, our point is that there exists at least some learners
whose performance can be dramatically improved by 
at least one simple optimization scheme like DE. In data mining and machine learning field,
grid search combined with cross-validation is still de facto practice for parameter tuning.
How does grid search compare with DE in terms of running time/evaluations and performance.
Furthermore, besides DE, there 're many other evolutionary algorithms out there, can we
find a even better tuner than DE? At the same time, DE has its own magic parameters as well, 
which will definitely change the behavior of DE as a tuner. In this view, we have
to use another tuner to tune DE and then to tune Learners.Perhaps a better approach might be
to dispense with the separation of ``optimizer'' and ``learner'' and combine them both
into one system that learns how to tune itself as it executes.

\subsection{Improve Tuning Performance}
When any supervised learning algorithm is applied, we assume that the data will use 
to fit the learner come from the same source as the testing (predicting) data. In other
words, training data and testing data should be correlated. 
e
In this paper, we
design the training, tuning and testing data as the projects data from the same project but
released chronologically. We were tying to make the tuning and testing data much
correlated.However,  we still observed some negative results from tables \tab{precisionbars} and \tab{fbars}.
For example, both tuned WHERE and tuned Random Forests get decreased scores than default learners in {\it xercesV1}
experiment for precision goal. The same pattern can also be observed in table \tab{fbars} for F scores.
Furthermore, in figure \ref{fig:deltas}, we recognize that tuning learners don't have very big
improvements($>= 30$ percentage points) over default learners in $\frac{8}{17}$ of data sets. Because of those 8 data sets,
the impacts of parameter tuning for defect prediction are not amazing good overall. The challenge here is
can we improve tuning performance in those data sets? 

Recall the statement we made at the beginning of this section, tuning and testing data have to be correlated.
Even though they are all from the same project, we still can't make sure whether they're strongly
correlated. What we can do here is to apply some heuristics to select the more correlated data from training set as tuning data based
on some properties of testing data without using label information. Up to now, the basic idea is to cluster
training and testing data to find nearest neighbours
by using distance metrics. Specifically, the ideas are described as following:\\
Idea A
\bi
\item cluster the training and testing data, separately.
\item find some distance measure between training clusters and testing clusters.
\item run all our tuning methods on the closest training data clusters.
\item run the testing clusters using the tunings generated by their nearest training clusters.
\ei
Idea B
\bi
\item make labels for testing and training data.
\item cluster all training and testing data together.
\item select those data that sit close to testing data as training(or tuning) data.
\item run tuning methods on the selected training data
run testing using the tunings generated by selected training data.
\ei


\subsection{Tune more learners}
The results of this paper as well as \cite{tantithamthavorn2016automated} show that tuning
can improve performance of defect predictor on most data
sets. However, whether other software analytics tasks like
duplicate bug reports detection \cite{sun2010discriminative,jalbert2008automated,alipour2013contextual,nguyen2012duplicate}, and bug reports classification \cite{antoniol2008bug,zanetti2013categorizing,lamkanfi2011comparing,tian2013drone} can be improved or not are unclear. It's worthy to explore more tuning practices in such topics in software engineering.

Recent research in software engineering community has examined and applied deep learning to address problems
, like code suggestion \cite{white2015toward} and localize buggy files \cite{lam2015combining}, which demonstrates its effectiveness at real software engineering tasks compared to state of the art method. However, deep learning networks usually include multiple levels of nonlinear transformations\cite{bengio2009learning}. Models like RNNs have many number of hyper parameters to control the whole system, like the number of hidden layer, the learning rate, and the amount of regularization, etc. However, most deep learning work strongly relies on engineering tricks that are difficult to be repeated and studied by others, apart from the authors themselves\cite{zhou2014big}. Up to now, there's no research on tuning deep learning parameters using software engineering data sets. There are several open questions as followings:
\bi
\item How to efficiently tune deep learning? 
\item Is the train-tune-test framework still available for deep learning given the long running time of deep learning in current literature?
\ei


\section*{Acknowledgments}
The work has partially funded by a National Science Foundation CISE CCF award \#1506586.
 
\vspace*{0.5mm}
 
 
\bibliographystyle{ieeetr}
% \bibliographystyle{elsarticle-num}

\bibliography{tuningpredictor}  

\balance

\end{document}
 
% \subsection{Implications}

% time for an end to era of data mining in se? moving on to a new phase of learning-as-optimization

% 1) learning is actually an optimization tasks (e.g. see fig2 of  learners climbing the roc curve hill in http://goo.gl/x2EaAm)

% 2) our learners are all contorted to do some tasks X (e.g. minimize expected value of entropy), then we assess them on score Y (recall). which is nuts. maybe we should build the goal predicate into the learner (e.g http://menzies.us/pdf/10which.pdf) 

% 3) given 1 + 2, maybe the whole paradigm of optimizing param selection is wrong. maybe what we need is a library of bees buzzing around making random choices (e.g. about descritziation) which other bees use, plus their own random choices (e.g. max depth of tree learned from discretized data) which is used by other bees, plus their own random choices (e.g. business users reading the models).  the funky thing here is that it can take some time before some of the bees (the discretizers) get feedback from the community of people using their decision (the tree learners). 




