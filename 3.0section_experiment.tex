\section{Experimental Design}\label{sect:design}
 

\subsection{Data Sets}\label{sect:dataa}

Our defect data comes from the PROMISE repository \footnote{http://openscience.us/repo}
and pertains to 
open source Java systems defined in terms of \tab{ck}:  {\it ant}, {\it camel}, {\it ivy}, {\it jedit}, {\it log4j}, {\it lucene},
{\it poi}, {\it synapse}, {\it velocity} and {\it xerces}. 

An important principle in data mining is not to test on the data used
in training.  There are many ways to design a experiment that satisfies this principle.
Some of those methods have  limitations; e.g.
{\em leave-one-out} is too slow for large data sets and
{\em cross-validation} mixes up older and newer data  (such that
data from the {\em past} may be used to test on {\em future data}).

To avoid these problems, we used an incremental learning approach. The following
experiment ensures that the training data was created at some time before the test
data.
For this experiment, we use data sets with at least three  
consecutive releases  (where release $i+1$ was built after release $i$). When tuning a learner,
\bi 
% \item {\em Tuned learner}: the {\em first} release is used  on line 24 of Algorithm~\ref{alg:DE} to
%   build some model using some the tunings found in some {\em Candidate}. The {\em second} release was used on line 25 of Algorithm~\ref{alg:DE} to 
%   test the model found on line 24. The {\em third} release was used as a testing data to gather the performance statistics
%   reported below from the best model found by DE.
% \item {\em untuned learner}:



\item The {\em first} release was used  on line 24 of Algorithm~\ref{alg:DE} to
   build some model using some the tunings found in some {\em Candidate}.
 \item The {\em second} release was used on line 25 of Algorithm~\ref{alg:DE} to 
   test the candidate model found on line 24.
   \item Finally the {\em third} release was used to gather the performance statistics
   reported below from the best model found by DE.
 \ei
 
 To be fair for the untuned learner, the {\em first} and {\em second} releases used in tuning experiments
 will be combined as the training data to build a model. Then the performance of this untuned learner 
 will be evaluated by the same {\em third} release as in the tuning experiment.
 
% This approach ensures   all treatments 
% are assessed on the same tests. Note that we did consider
% one other experimental design but rejected it for reasons of
% internal validity (see \tion{construct}).

Some data sets have more than three releases and, for those data, we could run more
 than one experiment. For example, {\em ant} has five versions in PROMISE so
 we ran three experiments called V0,V1,V2:
 \bi
 \item AntV0: first, second, third = versions 1, 2, 3
 \item AntV1: first, second, third = versions 2, 3, 4
 \item AntV2: first, second, third = versions 3, 4, 5
 \ei 
These data sets are displayed in \tab{data1}.

% As an aside, an alternate experimental design would be to 
% learn a baseline learner from the first {\em and} second release
% instead of, as shown above,  just the first release. On the one hand,
% this would mean that the baseline could be learned from more data.
% On the other hand, this adds a conflation to our experimental design
% since the optimizer uses the second release for pruning, not growing a data set.  Happily, from 
% piror work~\cite{Menzies:2008aa} we know that defect predictors usually {\em saturates} (i.e.
% does not generate better predictors) after 100 examples, which is a number smaller than all our first release data sets. Hence, their would
% be little value in generating the baselines using the first and second
% releases. 

 \subsection{Data Miners}
 
There are several ways to make defect predictors
using  CART~\cite{brieman00}, Random Forest~\cite{breiman84}, 
 WHERE~\cite{menzies2013local} and LR (logistic regression).
For this study, we use CART, Random Forest and LR versions  from 
SciKitLearn~\cite{scikit-learn} and
WHERE, which is available from
github.com/ai-se/where. 
 We use  these algorithms for the following reasons.
 
CART and Random Forest were mentioned in
a recent IEEE TSE paper by Lessmann et al.~\cite{lessmann2008benchmarking} that compared 22  
learners for  defect prediction.
That study ranked  CART  worst  and Random Forest as best.
In a demonstration of the impact of tuning,
this paper shows  we can {\em refute} the conclusions of  Lessmann et al.
in the sense that, after tuning,
CART
performs just as well as
 Random Forest.

LR was  mentioned by Hall et al.~\cite{hall11}
as usually being as good or better as more complex learners (e.g.
Random Forest). In a finding that endorses the Hall et al. result,
we show that untuned LR performs better than 
untuned Random Forest (at least, for the data sets studied here). However,
we will show that tuning raises doubts about the optimality of the
Hall et al. recommendation.

Finally,  this
 paper uses WHERE since, as shown below,
it offers the new perspective that tuning could change the
most important features selected by the learner.
  
%%%%%%%%%%%%%%%% list of parameters%%%%%%%%%%%%%%%%%%%%%
\renewcommand\arraystretch{1.2}
\begin{table*}[t!]
\scriptsize
  \centering
	\begin{tabular}{|c|c|c|c|l|}
	\cline{1-5}
	\begin{tabular}[c]{@{}c@{}}Learner Name\end{tabular} & Parameters & Default &\begin{tabular}[c]{@{}c@{}}Tuning\\ Range\end{tabular}& 
\multicolumn{1}{c|}{Description} \\ \hline
 	\multirow{8}{*}{\begin{tabular}[c]{@{}c@{}}Where-based\\ Learner\end{tabular}}
%  	WHERE-based  Learner} 
	& threshold & 0.5 &[0.01,1]& The value to determine defective or not .\\ \cline{2-5} 
	& infoPrune & 0.33 &[0.01,1]& The percentage of features to consider for the best 
split to build its final decision tree. \\ \cline{2-5} 
	 & min\_sample\_split & 4& [1,10]& The minimum number of samples required to split an internal node of
its final  decision tree. \\ \cline{2-5} 
	 & min\_Size & 0.5 &[0.01,1]& \begin{tabular}[c]{@{}l@{}}Finds min\_samples\_leaf 
in the initial clustering tree using ${n\_samples}^ {min\_Size}$.
\end{tabular} \\ \cline{2-5} 
    & wriggle & 0.2 &[0.01, 1] & The threshold to determine which branch in  the initial clustering tree to be pruned\\ \cline{2-5}
	 & depthMin & 2 & [1,6]&The minimum depth of the initial clustering tree below which no pruning for the
clustering tree. \\ \cline{2-5} 
	 & depthMax & 10 &[1,20]& The maximum depth of the initial clustering tree. \\ \cline{2-5} 
	 & wherePrune & False &T/F& Whether or not to prune the initial clustering tree. \\ \cline{2-5}
	 & treePrune & True &T/F& Whether or not to prune the final decision tree. \\ \cline{2-5} 
\hline
\multirow{5}{*}{CART} & threshold & 0.5 &[0,1]& The value to determine defective or not. \\ \cline{2-5} 
	 & max\_feature & None &[0.01,1]& The number of features to consider when looking for the best 
split. \\ \cline{2-5} 

	 & min\_sample\_split & 2 &[2,20]& The minimum number of samples required to split an 
internal node. \\ \cline{2-5} 
	 & min\_samples\_leaf & 1 & [1,20]&The minimum number of samples required to be at a leaf 
node. \\ \cline{2-5} 
     & max\_depth & None & [1, 50]& The maximum depth of the tree. \\
     \cline{1-5}  
       \multirow{5}{*}{\begin{tabular}[c]{@{}c@{}}Random \\ Forests\end{tabular}}  & threshold & 0.5 & [0.01,1] & The value to determine defective or not. \\ 
\cline{2-5} 
	 & max\_feature & None &[0.01,1]& The number of features to consider when looking for the best 
split. \\ \cline{2-5} 
	 & max\_leaf\_nodes & None &[1,50]& Grow trees with max\_leaf\_nodes in best-first fashion. \\ \cline{2-5} 
	 & min\_sample\_split & 2 &[2,20]& The minimum number of samples required to split an 
internal node. \\ \cline{2-5} 
	 & min\_samples\_leaf & 1 &[1,20]&The minimum number of samples required to be at a leaf 
node. \\ \cline{2-5} 
	 &  n\_estimators & 100 & [50,150]&The number of trees in the forest.\\ \cline{2-5}
	 \hline 
Logistic Regression&\multicolumn{4}{c|}{This study uses untuned LR in order to check
a conclusion of~\cite{hall11}. }\\\hline

	\end{tabular}
    \caption {List of parameters tuned by this paper.}
\label{tab:parameters}
\end{table*}

\subsection{ Learners' Tunings}


Our learners use the tuning parameters of \tab{parameters}. This section describes those parameters.
The default parameters for CART and Random Forest are set by 
the SciKitLearn authors and the
default parameters for WHERE-based learner are set via our own expert judgement.
When we say a learner is used ``off-the-shelf'', we mean
that they use the defaults shown in \tab{parameters}. 

As to the value of those defaults, it could be argued that these defaults are 
not the best  parameters needed for practical defect prediction.
That said,  prior to this paper, two things were true:
\bi
\item 
Many data scientists in SE use the standard defaults
in their data miners, 
without   tuning (e.g.~\cite{me07b,Moser:2008,herzig2013predicting,zimmermann2007predicting}).
\item
The effort involved to adjust those tunings seemed so onerous, that
many researchers in this field were content to take our prior advice
of ``do not tune... it is just too hard''~\cite{me15:book1}.
\ei
As to why we used the "Tuning Range" shown in \tab{parameters}, and not some other ranges,
we note that (1)~those ranges included the defaults; (2)~the results shown below
show that by exploring those ranges,   we achieved large gains in the performance of our defect predictors.
This is not to say that {\em larger} tuning ranges might not result in {\em greater} improvements.
However, for the goals of this paper (to show that some tunings do matter), exploring
just these ranges shown in \tab{parameters} will suffice.


% As to the details of these learners, LR is a parametric
% modeling approach. Given $f = \beta_0 + \sum_i\beta_ix_i$,
% where $x_i$ is some measurement in a data set, and $\beta_i$
% is learned via regression, LR
% converts that into a function $0 \le g \le 1$
% using $g=1/\left(1+e^{- f}\right)$. This function reports how much
% we believe in a particular class. 

CART, Random Forest, and WHERE-based learners are all  tree learners that divide a data set, then recur
on each split.
All these learners
generate numeric predictions which are converted
into binary ``yes/no'' decisions via \eq{yesno}.

\begin{equation}\label{eq:yesno}\scriptsize
\mathit{inspect}= \left\{
\begin{array}{ll}
d_i \ge T \rightarrow \mathit{Yes}\\
d_i <   T \rightarrow \mathit{No} ,
\end{array}\right.
\end{equation}
The splitting process is controlled by numerous tuning parameters.
If data contains more than {\em min\_sample\_split}, then a split is attempted.
On the other hand, if a split contains no more than {\em min\_samples\_leaf}, then the recursion stops. CART and Random Forest use a 
user-supplied constant for this parameter while
WHERE-based learner firstly computes this parameter $m$={\em min\_samples\_leaf} from the size of the data
sets via  $m=\mathit{size}^\mathit{min\_size}$ to build an initial  clustering tree.
Note that WHERE builds {\em two} trees: the initial clustering tree (to find similar sets of data)
then a final decision tree (to learn rules that predict for each similar cluster)\footnote{A
frequently asked question is why does WHERE build two trees--
would not   a single tree suffice? The answer is, as shown below,  tuned WHERE's twin-tree approach 
generates very precise predictors.}.
The tuning parameter  {\em min\_sample\_ split } controls the construction of the
final decision tree (so, for WHERE-based learner,
{\em min\_size} and {\em min\_sample\_split} are the parameters to be tuned).

These learners use different techniques to explore the splits:
\bi
\item
CART finds the attributes whose ranges contain rows with least variance in the number
of defects\footnote{If an attribute ranges $r_i$ is found in 
$n_i$ rows each with a  defect count variance of $v_i$, then CART seeks the attributes
whose ranges minimizes $\sum_i \left(\sqrt{v_i}\times n_i/(\sum_i n_i)\right)$.}.
\item
Random Forest    divides data like CART then  builds $F>1$  trees,
each time using some random subset of
the attributes. 
\item
When building the initial cluster tree, WHERE projects the data on to a dimension it synthesizes from the raw data using
a process analogous to principle component analysis\footnote{
PCA  synthesises  new
attributes $e_i, e_2,...$
that extends across the dimension of greatest  variance in the data  with attributes $d$.  
This process  combines
redundant  variables into a smaller set of variables  (so $e \ll d$) since those
redundancies become (approximately) parallel lines
in $e$ space. For all such redundancies \mbox{$i,j \in d$}, we 
can ignore $j$ 
since effects that change over $j$ also
change in the same way over $i$.
PCA is also useful for skipping over noisy variables from $d$-- these
variables are effectively ignored since    they  do not contribute to the variance in the data.}.
WHERE  divides  at the median point of that projection.
On recursion,
this generates the initial clustering tree, the leaves of which are clusters of  very similar examples. After that, when building 
the final decision tree, WHERE pretends its clusters are ``classes'', then 
asks the InfoGain of the
Fayyad-Irani discretizer~\cite{FayIra93Multi}, to rank the attriubutes, where {\em infoPrune} is used.
WHERE's final decision tree generator then ignores everything except the top   {\em infoPrune} percent of the sorted
attributes.
\ei
Some tuning parameters are learner specific:
\bi
\item
{\em Max\_feature} is used by
CART and Random Forest to select the number of attributes
used to build one tree.
CART's default is to use all the attributes while 
Random Forest usually selects the square root of the number
of attributes.
\item
  {\em Max\_leaf\_nodes} is the upper bound on leaf notes generated in a 
  Random Forest.
\item {\em Max\_depth} is the upper bound on the depth of the CART tree.  
 \item
WHERE's  
tree generation will always split up to {\em depthMin} number of branches.
After that, WHERE will only split data if the mean performance scores of the two halves
is ``trivially small'' (where ``trivially small'' is set by the   {\em wriggle} parameter). 
\item
WHERE's   {\em tree\_prune} setting controls how   
WHERE prunes back superflous parts of the final decision tree. 
If a decision sub-tree and its parent have the same 
majority cluster
(one that occurs most frequently), then if {\em tree\_prune} is enabled, we prune that decision sub-tree.
\ei


\subsection{Optimization Goals}

Recall from Algorithm~1 that we call differential evolution once for each
 optimization goal. This section lists those optimization goals.
Let $\{A,B,C,D\}$ denote the
true negatives, 
false negatives, 
false positives, and 
true positives
(respectively) found by a binary detector. 
Certain standard measures can be computed from
$A,B,C,D$, as shown below. Note that for $pf$, the {\em better} scores are {\em smaller}
while
for all other scores, the {\em better} scores are {\em larger}.

{\scriptsize\[
\begin{array}{ll}
pd=recall=&D/(B+D)\\
pf=&C/(A+C)\\ 
prec=precision=&D/(D+C) \\
F =&2*pd*prec/(pd + prec)
\end{array}
\]}

The rest of this paper explores tuning for {\em prec} and {\em F}. Our point is not that these are best or most important optimization goals.
Indeed, the list of ``most important'' goals is domain-specific
and we only explore these two to illustrate how conclusions can change dramatically
when moving from one goal to another.