
\section{Reliability and Validity}\label{sect:construct}


{\em Reliability} refers to the consistency of the results obtained
from the research.  For example,   how well independent researchers
could reproduce the study? To increase external
reliability, this paper has taken care to either  clearly define our
algorithms or use implementations from the public domain
(SciKitLearn). Also, all the data used in this work is available
on-line in the PROMISE code repository and all our algorithms
are on-line at github.com/ai-se/where.



{\em External validity} checks if the results are of relevance
for other cases, or can be generalized from samples
to populations.  
The examples of this paper  only relate to precision, recall, and the F-measure
but the general principle (that the search bias changes the search conclusions)  holds for any set of goals. 
Also,
the tuning results shown here only came from one  software analytics task 
(defect prediction from static code attributes).
There are many other kinds of software analytics tasks 
(software development effort estimation, social network mining,
detecting duplicate issue reports, etc) and the implication of this
study for those tasks is unclear. 
However,  those other tasks often use the same kinds of learners
explored in this paper so it is quite possible that
the conclusions of this paper apply to other SE analytics tasks as well. 

%That said, there exist some class of data mining papers for which
%tuning may not be required. Consider  Le Goues et al.'s 2012
%ICSE paper that used a evolutionary program to learn
%repairs to code~\cite{leGoues12}. The performance criteria
%in that paper was ``can we fix any of the known bugs?''. Note
%that this criteria is a ``{\em competency}'' statement, and
%not a ``{\em better than}'' statement (the difference being that
%one is 
%``can do'' and the other is ``can do better''). For such
%competency claims, tuning is not necessary. However, as soon
%as {\em better than} enters the performance criteria then this
%becomes a race between competing methods. In such a race,
%it is unfair to hobble one competitor with poor tunings.